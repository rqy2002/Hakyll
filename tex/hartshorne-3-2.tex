\documentclass{article}
\usepackage{xeCJK}
\usepackage{fontspec}
\usepackage{amsmath}
\usepackage{amsthm}
\usepackage{unicode-math}
\setmathfont{Libertinus Math}
\usepackage[a4paper, margin=2cm]{geometry}
\usepackage{enumitem}
\usepackage{tikz-cd}
\usepackage{hyperref}
\hypersetup{
  colorlinks = true,
  linkcolor = blue
}

\def\proofname{证明}

\newtheoremstyle{exercise}%
{3pt}{3pt}
{}%
{}%
{\bfseries}%
{.}%
{.2em}%
{}
\theoremstyle{exercise}
\newtheorem{exercise}{习题}[section]
\theoremstyle{theorem}
\newtheorem*{lemma*}{引理}
\theoremstyle{remark}
\newtheorem*{remark*}{注}

\def\ga{\mathfrak{a}}
\def\gp{\mathfrak{p}}
\def\gq{\mathfrak{q}}
\def\gm{\mathfrak{m}}
\def\gn{\mathfrak{n}}
\def\A{\mathbb{A}}
\def\P{\mathbb{P}}
\def\Z{\mathbb{Z}}
\def\F{\mathbb{F}}
\def\Q{\mathbb{Q}}
\def\R{\mathbb{R}}
\def\C{\mathbb{C}}
\def\bS{\mathbf{S}}
\def\Ab{\mathfrak{Ab}}
\def\id{\mathrm{id}}
\def\cO{\mathscr{O}}
\def\cK{\mathscr{K}}
\def\cF{\mathscr{F}}
\def\cG{\mathscr{G}}
\def\cH{\mathscr{H}}
\def\cI{\mathscr{I}}
\def\cR{\mathscr{R}}
\def\Sch{\mathfrak{Sch}}
\def\Rings{\mathfrak{Rings}}
\def\spto{\rightsquigarrow}
\def\red{\mathrm{red}}
\def\spp{\operatorname{sp}}
\def\Hom{\operatorname{Hom}}
\def\Spec{\operatorname{Spec}}
\def\Proj{\operatorname{Proj}}
\def\Supp{\operatorname{Supp}}
\def\coker{\operatorname{coker}}
\def\im{\operatorname{im}}
\def\dim{\operatorname{dim}}
\def\codim{\operatorname{codim}}
\def\height{\operatorname{height}}
\def\Frac{\operatorname{Frac}}
\def\trd{\operatorname{tr.d.}}
\def\leq{\leqslant}
\def\geq{\geqslant}
\def\Stacks#1{\href{https://stacks.math.columbia.edu/tag/#1}{Stacks Project #1}}

\begin{document}

这学期开始上盛茂的代数几何课, 于是做一做习题, 毕竟这也是作业.

本节已经做完.

\setcounter{section}{2}

\begin{exercise} \hfill
  \begin{enumerate}
    \item 令 $X = \A_k^1$ 为无限域 $k$ 上的仿射直线.
          令 $P, Q$ 为 $X$ 中两个闭点, $U = X - \{P, Q\}$.
          试证明 $H^1(X, \Z_U) \neq 0$.
    \item 考虑更一般的情形,
          令 $Y \subseteq X = \A_k^n$ 为一般位置的 $n + 1$ 个超平面的并.
          令 $U = X - Y$. 试证明 $H^n(X, \Z_U) \neq 0$.
          因此 (2.7)\footnote{
          Grothendieck 消失定理, $n$ 维 Noether 空间的超过 $n$ 阶上同调消失.}
          的结论已经最佳.
  \end{enumerate}
\end{exercise}

\begin{proof} \hfill
  \begin{enumerate}
    \item 设 $Y = \{ P, Q \}$, 记 $i \colon Y \hookrightarrow X$.
          令 $\Z_Y = i_*(\Z |_Y)$, 则显然有正合列
          \[
          0 \to \Z_U \to \Z \to \Z_Y \to 0.
          \]
          但 $\Z_Y$ 同构于 $Z$ 在 $P$ 和 $Q$ 上的摩天楼层的直和,
          因此 $\Gamma(X, \Z_Y) = \Z^2$. 所以对上述正合列取 $\Gamma(X, -)$ 得
          \[
          0 \to 0 \xrightarrow{1} \Z \xrightarrow{(1, 1)} \Z^2 \to H^1(X, \Z_U).
          \]
          由于  $\Z \to \Z^2$ 不是满射, 即知 $H^1(X, \Z_U) \neq 0$.
    \item 记 $Y_{n, k}$ 为 $\A^{n+1}$ 里 $k$ 个一般位置超平面的并集构成的子空间,
          $U_{n+1, k}$ 为 $Y_{n, k}$ 在 $\A^{n+1}$ 里的补集.
          则有 $\A^{n+1}$ 上的正合列 $0 \to \Z_{U_{n+1, k}} \to \Z \to \Z_{Y_{n, k}} \to 0$,
          其长正合列便给出 $H^{i+1}(\A^{n+1}, \Z_{U_{n+1, k}}) \cong H^i(\A^{n+1}, \Z_{Y_{n, k}}) \cong H^i(Y_{n, k}, \Z)$ ($i > 0$),
          以及 $H^1(A^{n+1}, \Z_{U_{n+1, k}}) = \coker(\Z \to \Gamma(Y_{n, k}, \Z)) = \begin{cases} \Z^{k-1} & n = 0, \\ 0 & n > 0. \end{cases}$

          固定 $n, k$. 则 $Y_{n, k}$ 中前 $k-1$ 个超平面的并可以记作 $P \cong Y_{n, k-1}$,
          最后一个超平面记作 $Q \cong \A^n$, 则 $Q \setminus P \cong U_{n, k-1}$. 从而又有正合列
          $0 \to \Z_{U_{n, k-1}} \to \Z_{Y_{n, k}} \to \Z_{Y_{n, k-1}} \to 0$.
          由此又有长正合列
          \begin{align*}
          \cdots &\to H^{i-1}(Y_{n, k-1}, \Z_{Y_{n, k-1}})
          \to H^i(\A^n, \Z_{U_{n, k-1}})
          \to H^i(Y_{n, k}, \Z_{Y_{n, k}}) \\
          &\to H^i(Y_{n, k-1}, \Z_{Y_{n, k-1}})
          \to H^{i+1}(\A^n, \Z_{U_{n, k-1}})
          \to \cdots.
          \end{align*}
          而 $H^i(\A^n, U_{n, k-1}) \cong \begin{cases} H^{i-1}(Y_{n-1, k-1}, \Z_{Y_{n-1, k-1}}) & i > 1, \\ \Z^{k-1} & n=1, i=0, \\ 0 & n>1, i=0. \end{cases}$

          记 $A_{n, k}^i = \begin{cases} H^i(Y_{n, k}, \Z_{Y_{n, k}}) \otimes \Q & i > 0, \\ 0 & i = 0. \end{cases}$. 上述长正合列通过 $\otimes \Q$ 化为
          \[
          \cdots \to A_{n, k-1}^{i-1} \to A_{n-1, k-1}^{i-1} \to A_{n, k}^i \to A_{n, k-1}^i \to A_{n-1, k-1}^i \to \cdots.
          \]
          在 $n = 1, i = 1$ 时有特例: $0 \to \Q \to \Q^{k-1} \to A_{1, k}^1 \to A_{1, k-1}^1 \to \dots$.

          我们需要证明 $H^n(\A^n, U_{n, n+1}) \cong A_{n-1, n+1}^{n-1} \neq 0$, 或者说 $A_{n, n+2}^n \neq 0 (n > 0)$.

          为此, 我们证明: 对任意 $n > 0$ 及任意 $1 \leq k \leq n + 1$, 有 $A_{n, k}^n = A_{n, k}^{n-1} = 0$;
          而对 $k = n + 2$, 有 $A_{n, n+2}^n \neq 0$.
          对 $(n, k)$ 字典序归纳:
          \begin{itemize}
            \item 若 $k = 1$, 则 $Y_{n, 1} \cong \A^n$, 而 $\Q$ 是其上的松层, 因此其上同调都消失.
            \item 若 $n = 1, k = 2$, 则 $0 \to \Q \to \Q \to A_{1, 2}^1 \to 0$. 因此 $A_{1, 2}^1 = 0$.
            \item 若 $n = 1, k = 3$, 则由正合列 $0 \to \Q \to \Q^2 \to A_{1, 3}^1 \to 0$, 得 $A_{1, 3}^1 \cong \Q$.
            \item 若 $n > 1, 1 < k \leq n + 1$,
                  则由正合列 $A_{n-1, k-1}^{i-1} \to A_{n, k}^i \to A_{n, k-1}^i$,
                  代入 $i = n, n-1$ 由归纳假设即证.
            \item 若 $n > 1, k = n + 2$, 则由正合列 $A_{n, n+1}^{n-1} \to A_{n-1, n+1}^{n-1} \to A_{n, n+2}^n \to A_{n, n+1}^n$
                  及归纳假设: $A_{n, n+1}^{n-1} = A_{n, n+1}^n = 0$ 即得 $A_{n, n+2} \cong A_{n-1, n+1}^{n-1} \neq 0$ (亦为归纳假设).
          \end{itemize}
          事实上这说明总有 $A_{n, n+2}^n \cong A_{1, 3}^1 \cong \Q \neq 0$.
          因此 $H^n(\A^n, \Z_{U_{n, n+1}}) \neq 0$.

          \textit{不知道是否有 $H^1(Y_{1, 3}, \Z) \cong \Z$. 这似乎需要显式把 $\delta$ 映射算出来.}
          \qedhere
  \end{enumerate}
\end{proof}

\begin{exercise}
  令 $X = \P_k^1$ 为代数闭域 $k$ 上的射影直线.
  试证明第二章习题 1.21d 中的正合列
  \[ 0 \to \cO \to \cK \to \cK / \cO \to 0 \]
  是 $\cO$ 的松消解. 从而由此习题 e 得出对任意 $i > 0$ 总有 $H^i(X, \cO) = 0$.
\end{exercise}

\begin{proof}
  由于 $\cK$ 是 $K$ 常值层, 其显然松.
  而由习题  II, 1.21d, $\cK / \cO \cong \sum_{P \in X} i_P(K / \cO_p)$,
  也是松层.

  由上同调长正合列即知对 $i > 2$ 都有 $H^i(X, \cO) = 0$,
  而 $H^1(X, \cO) = \coker(\Gamma(X, \cK) \to \Gamma(X, \cK / \cO))$.
  由 II, 1.21e, $\Gamma(X, \cK) \to \Gamma(X, \cK / \cO)$ 满. 因此 $H^1(X, \cO) = 0$.
\end{proof}

\begin{exercise}[子集支撑的上同调]
  令 $X$ 为拓扑空间, $Y$ 为闭子集, $\cF$ 为 Abel 群层.
  令 $\Gamma_Y(X, \cF)$ 表示 $\cF$ 里支在 $Y$ 上的截面的群.
  \begin{enumerate}[label=(\arabic*)]
    \item 证明 $\Gamma_Y(X, -)$ 是 $\Ab(X) \to \Ab$ 的左正合函子.

          记 $\Gamma_Y(X, -)$ 的右导出函子为 $H_Y^i(X, -)$.
          它们是 \emph{$X$ 的支在 $Y$ 上的上同调群}.
    \item 若 $0 \to \cF' \to \cF \to \cF'' \to 0$ 是层的正合列, $\cF'$ 松,
          试证明
          \[ 0 \to \Gamma_Y(X, \cF') \to \Gamma_Y(X, \cF) \to \Gamma_Y(X, \cF'') \to 0 \]
          正合.
    \item 证明若 $\cF$ 松, 则对任意 $i > 0$ 有 $H_Y^i(X, \cF) = 0$.
    \item 若 $\cF$ 松, 试证明
          \[ 0 \to \Gamma_Y(X, \cF) \to \Gamma(X, \cF) \to \Gamma(X - Y, \cF) \to 0 \]
          正合.
    \item 记 $U = X - Y$. 证明对任意 $\cF$, 有上同调长正合列
          \begin{align*}
            0 & \to H_Y^0(X, \cF) \to H^0(X, \cF) \to H^0(U, \cF|_U) \\
              & \to H_Y^1(X, \cF) \to H^1(X, \cF) \to H^1(U, \cF|_U) \\
              & \to H_Y^2(X, \cF) \to \cdots.
          \end{align*}
    \item \emph{切除}. 令 $V$ 是 $X$ 的某个包含 $Y$ 的开子集.
          则有对 $i, \cF$ 自然的同构
          \[ H_Y^i(X, \cF) \cong H_Y^i(V, \cF|_V). \]
  \end{enumerate}
\end{exercise}

\begin{proof} \hfill
  \begin{enumerate}[label=(\arabic*)]
    \item $\Gamma_Y(X, -)$ 的函子性显然.
          设 $0 \to \cF' \xrightarrow{f} \cF \xrightarrow{g} \cF'' \to 0$ 正合.
          显然 $\Gamma_Y(X, \cF') \to \Gamma_Y(X, \cF)$ 是单射.

          而若 $s \in \Gamma_Y(X, \cF)$ 且 $g(s) = 0$,
          则由 $\Gamma(X, -)$ 左正合性知存在 $s' \in \Gamma_Y(X, \cF')$ 使得 $f(s') = s$.
          由于 $\cF' \to \cF$ 单, 其在茎上也单. 于是 $s' \in \Gamma_Y(X, \cF')$.

          综上, $0 \to \Gamma_Y(X, \cF') \to \Gamma_Y(X, \cF) \to \Gamma_Y(X, \cF'')$
          正合.
    \item 设 $\cF, \cF', \cF''$ 同上, 设  $s'' \in \Gamma_Y(X, \cF'')$.
          由于 $\cF'$ 松, 存在 $s \in \Gamma(X, \cF)$ 使得 $g(s) = s''$.

          记  $U = X - Y$. 由于 $\cF'$ 松,
          $0 \to \Gamma(U, \cF') \to \Gamma(U, \cF) \to \Gamma(U, \cF'') \to 0$ 正合.
          而 $g(s|_U) = g(s)|_U = 0$. 因此存在  $s'_0 \in \Gamma(U, \cF')$
          使得 $f(s'_0) = s|_U$. 再次由松性, 存在 $s' \in \Gamma(X, \cF')$
          使得 $s|_U = s'_0$. 因此立即知道 $s' - f(s) \in \Gamma_Y(X, \cF)$
          且其像为 $s''$. 因此
          \[ 0 \to \Gamma_Y(X, \cF') \to \Gamma_Y(X, \cF) \to \Gamma_Y(X, \cF'') \to 0 \]
          正合.
    \item 取内射层 $\cI$ 与单射 $\cF \to \cI$. 由于内射层松, $\cI$ 与 $\cI / \cF$ 都松.
          因此由长正合列 , $H_Y^n(X, \cF) \cong H_Y^{n-1}(X, \cI / \cF)$
          且 $H_Y^1(X, \cF) = 0$.
          因此对 $n$ 归纳立知 $H_Y^n(X, \cF) = 0$.
    \item 记
          $\Gamma_Y(X, \cF) \xrightarrow{i} \Gamma(X, \cF)
          \xrightarrow{p} \Gamma(X - Y, \cF) \to 0$
          按定义, $i$ 是单射, $p$ 是满射, 且 $pi = 0$.
          而若 $s \in \Gamma(X, \cF), p(s) = 0$, 则按定义 $s$ 支在 $Y$ 上,
          即 $s \in \im i$. 因此
          \[ 0 \to \Gamma_Y(X, \cF) \to \Gamma(X, \cF) \to \Gamma(X - Y, \cF) \to 0 \]
          正合.
    \item 取 $\cF$ 的内射消解 $\{\cI^\bullet\}$.
          由于内射模都松, 由上一个命题得知有链复形的正合列
          \[
          0 \to \Gamma_Y(X, \cI^\bullet)
          \to \Gamma(X, \cI^\bullet)
          \to \Gamma(U, \cI^\bullet|_U) \to 0
          \]
          而 $\cI^\bullet|_U$ 也松, 因此也是 $\cF|_U$ 的 $\Gamma(U, -)$-零调消解.
          取上述链复形短正合列对应的长正合列即为
          \begin{align*}
            0 & \to H_Y^0(X, \cF) \to H^0(X, \cF) \to H^0(U, \cF|_U) \\
              & \to H_Y^1(X, \cF) \to H^1(X, \cF) \to H^1(U, \cF|_U) \\
              & \to H_Y^2(X, \cF) \to \cdots.
          \end{align*}
    \item 首先, 我们有自然同构 $\Gamma_Y(X, \cF) \cong \Gamma_Y(V, \cF|_V)$,
          或 $\Gamma_Y(X, \cF) \cong \Gamma_Y(V, \cF)$.
          这个同构是显然的: 我们有前者到后者的限制映射, 而其根据层公理显然双射.

          现在设 $\cI^\bullet$ 是 $\cF$ 的内射消解,
          则 $\Gamma_Y(X, \cI^\bullet) \cong \Gamma_Y(V, \cI^\bullet|_V)$.
          由于 $\cI^\bullet|_V$ 也是 $\cF|_V$ 的松层消解,
          其上同调也给出 $H_Y^i(V, \cF|_V)$. 因此上述自然同构就给出了上同调的自然同构.
          \qedhere
  \end{enumerate}
\end{proof}

\begin{exercise}[Mayer--Vietoris 正合列]
  令 $Y_1, Y_2$ 为 $X$ 的两个闭集. 则有支集上同调的长正合列
  \begin{align*}
    \cdots & \to H^i_{Y_1 \cap Y_2}(X, \cF)
              \to H^i_{Y_1}(X, \cF) \oplus H^i_{Y_2}(X, \cF)
              \to H^i_{Y_1 \cup Y_2}(X, \cF) \\
            & \to H^{i+1}_{Y_1 \cap Y_2}(X, \cF) \to \cdots.
  \end{align*}
\end{exercise}

\begin{proof}
  记 $i_1, i_2$ 为 $\Gamma_{Y_1 \cap Y_2}(X, \cF)$ 到
  $\Gamma_{Y_1}(X, \cF)$ 或 $\Gamma_{Y_2}(X, \cF)$ 的自然嵌入,
  $j_1, j_2$ 为两者到 $\Gamma_{Y_1 \cup Y_2}(X, \cF)$ 的自然嵌入.
  设 $\cF$ 松, 下面证明
  \[
    0 \to \Gamma_{Y_1 \cap Y_2}(X, \cF)
    \xrightarrow{i_1 + i_2} \Gamma_{Y_1}(X, \cF) \oplus \Gamma_{Y_2}(X, \cF)
    \xrightarrow{j_1 - j_2} \Gamma_{Y_1 \cup Y_2}(X, \cF) \to 0
  \]
  正合.

  \begin{itemize}
    \item 显然 $i_1 + i_2$ 是单射, 且 $(j_1 - j_2)(i_1 + i_2) = 0$.
          且若 $(s_1, s_2) \in \Gamma_{Y_1}(X, \cF) \oplus \Gamma_{Y_2}(X, \cF)$
          使得 $j_1(s_1) - j_2(s_2) = 0$, 即 $s_1 - s_2 = 0$,
          必有 $s_1 = s_2 \in \Gamma_{Y_1 \cap Y_2}(X, \cF)$,
          即 $(s_1, s_2) \in \im(i_1 + i_2)$.
          因此 $\ker(j_1 - j_2) = \im(i_1 + i_2)$.
    \item 设 $\cF$ 松. 记 $U_1 = X - Y_1, U_2 = X - Y_2$.
          若 $s \in \Gamma_{Y_1 \cup Y_2}(X, \cF)$,
          令 $t \in \Gamma(U_1 \cup U_2, \cF)$
          为 $s|_{U_1}$ 和 $0|_{U_2}$ 的粘接.
          由于 $\cF$ 松, 存在 $s_2 \in \Gamma(X, \cF)$
          使得 $s_2|_{U_1 \cup U_2} = t$.

          由于 $s_2|_{U_2} = 0, s_1|_{U_1} = s|_{U_1}$,
          即得 $s_2 \in \Gamma_{Y_2}(X, \cF), s - s_2 \in \Gamma_{Y_1}(X, \cF)$.
          因此 $j_1 - j_2$ 为满射.
  \end{itemize}

  现在设 $\cF$ 为任意层. 取 $\cF$ 的内射消解 $\cI^\bullet$, 根据上述结论有正合列
  \[
    0 \to \Gamma_{Y_1 \cap Y_2}(\cI^\bullet, \cF)
    \xrightarrow{i_1 + i_2} \Gamma_{Y_1}(\cI^\bullet, \cF) \oplus \Gamma_{Y_2}(\cI^\bullet, \cF)
    \xrightarrow{j_1 - j_2} \Gamma_{Y_1 \cup Y_2}(\cI^\bullet, \cF) \to 0
  \]
  取其上同调长正合列
    \begin{align*}
      \cdots & \to H^i_{Y_1 \cap Y_2}(X, \cF)
               \to H^i_{Y_1}(X, \cF) \oplus H^i_{Y_2}(X, \cF)
               \to H^i_{Y_1 \cup Y_2}(X, \cF) \\
             & \to H^{i+1}_{Y_1 \cap Y_2}(X, \cF) \to \cdots.
    \qedhere
    \end{align*}
\end{proof}

\begin{exercise}
  设 $X$ 是 Zariski 空间 (II, 习题 3.17)\footnote{每个非空不可约闭集都有一一般点的 Noether 空间.}.
  令 $P \in X$ 为闭点, $X_P$ 为所有满足 $P \in \{Q\}^-$ 的点 $Q$ 构成的子集.
  称 $X_P$ 为 $X$ 在 $P$ 处的\emph{局部空间}, 配备诱导子空间拓扑.
  令 $j \colon X_P \to X$ 为包含映射; 对 $X$ 上的任意层 $\cF$, 记 $\cF_P = j^* \cF$.
  证明对任意 $i, \cF$ 都有
  \[
    H_P^i(X, \cF) \cong H_P^i(X_P, \cF_P).
  \]
\end{exercise}

下面的证明中需要这个引理:
\begin{lemma*}
  设 $X$ 为 Zariski 拓扑, $Y$ 为任意在一般化下封闭的子集, $j \colon Y \to X$ 为嵌入映射.
  设 $\cF$ 为 $X$ 上的层, $\cF_Y = j^*Y$. 则对 $Y$ 中任意开集 $U$, 有
  \[
    \cF_Y(U) \cong \varinjlim_{U \subset \tilde{U}} \cF(\tilde{U}),
  \]
  其中 $\tilde{U}$ 遍历 $X$ 的满足条件的开集.
  换句话说, $\cF$ 作为预层在 $Y$ 上的限制已经是层.
\end{lemma*}

\begin{proof}
  由于 $\cF_Y$ 事实上定义为 $U \mapsto \varinjlim_{U \subset \tilde{U}} \cF(\tilde{U})$ 的层化,
  立刻有 $\varinjlim_{U \subset \tilde{U}} \cF(\tilde{U})$ 到 $\cF_Y(U)$ 的映射.
  我们记此正向极限为 $\cF'_Y(U)$, 此映射为 $\varphi \colon \cF'_Y(U) \to \cF_Y(U)$.

  由于正向极限正合, $\cF'_Y(U) \to \prod_{x \in U} \cF_x$ 是单射. 因此易知 $\varphi$ 为单射.

  为证明 $\varphi$ 是满射, 也就是证明 $\cF_Y(U)$ 中任意截面 $s$ 都是 $U$ 附近的某个开集 $\tilde{U}$ 的某个截面的限制.
  按定义, 存在 $Y$ 的一族开覆盖 $U_i$ 使得 $s|_{U_i} \in \cF'_Y(U_i)$,
  即存在 $X$ 中开集 $\tilde{U_i} \supset U_i$ 以及 $s_i \in \cF(\tilde{U_i})$
  使得 $s_i|_{U_i} = s|_{U_i}$\footnote{这里混淆了记号, 事实上应该是 $s_i$ 在 $\cF'_Y(U_i)$ 中的像是 $s|_{U_i}$.};

  由于 $Y$ 拟紧, 可以设 $U_i$ 是有限开覆盖. 利用归纳法, 又可以规约到只有两个开集的情况.
  此时由于 $s_1|_{U_1 \cap U_2} = s_2|_{U_1 \cap U_2}$, 存在 $X$ 中的开集 $V$,
  使得 $U_1 \cap U_2 \subset V \subset \tilde{U_1} \cap \tilde{U_2}$,
  且 $s_1|_V = s_2|_V$.
  记 $N = (\tilde{U_1} \cap \tilde{U_2}) \setminus V$. 我们证明: $\overline{N} \cap Y = \varnothing$.
  若不然, 由于 $Y$ 对一般化封闭, 有 $\overline{N}$ 中的某个不可约分支的一般点 $\xi$ 属于 $Y$.
  因此 $\xi \in U_1 \cap U_2 \subset V$, 从而 $\xi \notin N$, 因而 $N \cap \{\xi\}^- = \varnothing$, 矛盾.

  从而, 若把 $U_1, U_2, V$ 分别改为 $U_1 \setminus \overline{N}, U_2 \setminus \overline{N}, V \setminus \overline{N}$,
  就有 $V = U_1 \cap U_2$. 因此可从 $s_1, s_2$ 拼出 $s_0 \in \cF(\tilde{U_1} \cup \tilde{U_2})$
  使得 $s_0|_U = s$. 也就是说 $\varphi$ 是满射. \qedhere
\end{proof}

\begin{proof}[习题 2.5 的证明]
  由于 Zariski 空间的开集对一般化封闭, 任意包含 $P$ 的开集都包含 $X_P$,
  且若 $Q \notin X_P$, 有 $P \in (X \setminus \{Q\}^-)$.
  所以 $X_P = \bigcap_{P \in U \subset X} U$, 其中 $U$ 取遍包含 $P$ 的开集.

  按习题 2.3 (6), 对任意包含 $P$ 的开集 $V$, 有 $H_P^i(X, \cF) \cong H_P^i(X_P, \cF_P)$.
  若 $V \supset W$, 取 2.3 (5) 的长正合列, 有映射
  \[
    \begin{tikzcd}
      \cdots \ar[r] & H_P^i(V, \cF|_V) \ar[r] \ar[d, "\cong"]
      & H^i(V, \cF|_V) \ar[r] \ar[d] & H^i(V \setminus \{P\}, \cF|_{V \setminus \{P\}}) \ar[r] \ar[d] & \cdots\\
      \cdots \ar[r] & H_P^i(W, \cF|_W) \ar[r]
      & H^i(W, \cF|_W) \ar[r] & H^i(W \setminus \{P\}, \cF|_{W \setminus \{P\}}) \ar[r] & \cdots
    \end{tikzcd}
  \]
  取正向极限即得正合列
  \[
    \cdots \to \varinjlim_{P \in V} H_P^i(V, \cF|_V)
    \to \varinjlim_{P \in V} H^i(V, \cF|_V)
    \to \varinjlim_{P \in V} H^i(V \setminus \{P\}, \cF|_{V \setminus \{P\}})
    \to \cdots.
  \]
  即
  \[
    \cdots \to H_P^i(X, \cF)
    \to \varinjlim_{P \in V} H^i(V, \cF)
    \to \varinjlim_{P \in V} H^i(V \setminus \{P\}, \cF)
    \to \cdots.
  \]
  此外, 在 $X_P$ 上还有长正合列
  \[
    \cdots \to H_P^i(X_P, \cF_P) \to H^i(X_P, \cF_P) \to H^i(X_P \setminus \{P\}, \cF_P) \to \cdots.
  \]
  因此只需证明有自然同构 $H^i(X_P, \cF_P) \cong \varinjlim_{P \in V} H^i(V, \cF)$
  及 $H^i(X_P \setminus \{P\}, \cF_P) \cong \varinjlim_{P \in V} H^i(V \setminus \{P\}, \cF)$.

  上面的引理事实上证明了在 $i = 0$ 时,
  \begin{align*}
    \Gamma(X_P, \cF_P) &\cong \varinjlim_{P \in V} \Gamma(V, \cF) \\
    \Gamma(X_P \setminus \{P\}, \cF_P) &\cong \varinjlim_{P \in V} \Gamma(V \setminus \{P\}, \cF).
  \end{align*}
  并且引理还有显然的推论: 若 $\cF$ 松, 则 $\cF_P$ 松 (因为 $\varinjlim$ 的正合性).
  因此立知: $H^i(X_P, -_P), H^i(X_P \setminus \{P\}, \cF_P)$ 都是可擦函子 (因为松层对于它们是零调对象).
  而 $\varinjlim H^i(V, -), \varinjlim H^i(V \setminus \{P\}, -)$ 显然也是可擦函子.
  因此由 $\delta$ 函子的万有性即得正合列的自然态射
  \[
    \begin{tikzcd}
      \cdots \ar[r] & H_P^i(X, \cF) \ar[r] \ar[d]
      & \varinjlim_{P \in V} H^i(V, \cF|_V) \ar[r] \ar[d, "\cong"]
      & \varinjlim_{P \in V} H^i(V \setminus \{P\}, \cF|_{V \setminus \{P\}}) \ar[r] \ar[d, "\cong"] & \cdots\\
      \cdots \ar[r] & H_P^i(X_P, \cF_P) \ar[r]
      & H^i(X_P, \cF_P) \ar[r] & H^i(X_P \setminus \{P\}, \cF_P|_{X_P \setminus \{P\}}) \ar[r] & \cdots
    \end{tikzcd}
  \]
  由五引理, 即知 $H_P^i(X, \cF) \to H_P^i(X_P, \cF_P)$ 亦为同构.
\end{proof}

\begin{exercise}
  令 $X$ 为 Noether 拓扑空间, $\{\cI_\alpha\}_{\alpha \in A}$ 为 $X$ 上内射层的有向系统. 证明 $\varinjlim \cI_\alpha$ 亦内射.
  [\emph{提示:} 首先证明层 $\cI$ 内射当且仅当对 $X$ 的任意开子集 $U$, $\Z_U$ 的任意子层 $\cR$, 以及任意态射 $f \colon \cR \to \cI$,
  其都可以扩张成 $\Z_U \to \cI$ 的映射.
  其次, 证明这样的 $\cR$ 都有限生成, 因此 $\cR \to \varinjlim \cI_\alpha$ 穿过某个 $\cI_\alpha$.]
\end{exercise}

\begin{proof}
  我们按照提示顺序证明. 首先证明层 $\cI$ 内射当且仅当对 $X$ 的任意开子集 $U$, $\Z_U$ 的任意子层 $\cR$, 以及任意态射 $f \colon \cR \to \cI$,
  其都可以扩张成 $\Z_U \to \cI$ 的映射.

  必要性显然. 考虑充分性. 设 $\cF$ 是 $X$ 上的层, $\cG \subseteq \cF$ 是其子层, $g \colon \cG \to \cI$ 是任意态射.
  我们希望证明 $g$ 可以延拓为 $\cF \to \cI$.
  记 $\Sigma = \{ (\cH, h) \mid \cG \subseteq \cH \subseteq \cF, h \colon \cH \to \cI, h|_{\cG} = g \}$,
  其上给显然的偏序. 由 Zorn 引理, $\Sigma$ 中有极大元 $(\cH, h)$.

  若 $\cH \neq \cF$, 取开集 $U$ 以及 $s \in \cF(U) \setminus \cH(U)$.
  记映射 $\varphi \colon \Z_U \to \cF, \varphi(t_V) = t_Vs|_V$. 记 $\cR = \varphi^{-1}(\cH)$.
  由假设, 映射 $h \circ \varphi \colon \cR \to \cI$ 可以延拓为 $\Z_U \to \cI$.
  此映射和 $h \colon \cH \to \cI$ 拼接为 $(\cH + \Z_Us) \to \cI$, 与 $(\cH, h)$ 极大性矛盾. 因此只可能 $\cH = \cF$.
  这里 $\cH + \Z_Us$ 定义为 $\cH \oplus \Z_U \to \cF$ 的像.

  \emph{Hartshorne 声称 $\cR$ 应该是有限生成的. 我想他大约想表达 $\cR$ 是 Noether 的, 即其子对象升链总稳定. 换句话说其希望说明 $\Z_U$ 是 Noether 的.}

  接下来我们证明 $\Z_U$ 是 Abel 群层里的 Noether 对象, 即其子对象升链总稳定.
  设 $\cF_1 \subseteq \cF_2 \subseteq \cdots \subseteq \Z_U$ 是 $\Z_U$ 的子对象的升链.
  设 $U$ 的不可约分支为 $U_1, \dots, U_n$, 则只需证明每个 $\cF_k|_{U_i}$ 稳定. 所以不妨设 $U$ 不可约.

  设 $V_k$ 为最大的使得 $\cF_k(V) \neq 0$ 的开集 (若 $\cF_k(V) = r\Z, \cF_k(V') = s\Z$, 显然有 $\cF_k(V \cup V') = \operatorname{lcm}(r, s)\Z$. 因此 $V_k$ 良定).
  则 $V_k$ 是开集升链, 从而稳定. 设其稳定到 $V$.
  则 $\cF_k(V)$ 是 $\Z$ 的子模升链, 其必定稳定.
  设其稳定到 $m\Z$. 按定义, $m > 0$.
  对 $m$ 的任意因子 $d$, 记 $C_k(d)$ 为 $d$ 在 $\Z_U / \cF_k$ 中的支集. 则 $C_k(d)$ 构成 (关于 $k$ 的) 闭集降链.
  因此每个都稳定. 也就是说, 在 $k$ 充分大的时候, $\cF_k \to \cF_{k+1}$ 在每个茎上都是同构. 这也就是说 $\cF_k$ 稳定.

  接下来, 设 $\cI_\alpha$ 为内射层的有向系统. 设 $U$ 是开集, $\cR$ 是 $\Z_U$ 的子层, $f \colon \cR \to \varinjlim \cI_\alpha$.
  由于 $\Z_U$ 是 Noether 对象, $f^{-1}(\cI_\alpha)$ 稳定. 也就是说存在 $\alpha$ 使得 $f$ 穿过 $\cI_\alpha$.
  因此由 $\cI_\alpha$ 的内射性, $f$ 可以延拓为 $\Z_U \to \cI_\alpha \to \varinjlim \cI_\alpha$. 综上, $\varinjlim \cI_\alpha$ 内射.
\end{proof}

\begin{exercise}
  令 $\bS^1$ 为圆, 配备通常的拓扑. 令 $\Z$ 为其上的常值层.
  \begin{enumerate}[label=(\arabic*)]
    \item 证明 $H^1(\bS^1, \Z) \cong \Z$ (用我们定义的(层)上同调).
    \item 现在令 $\cR$ 为 $\bS^1$ 上的连续实值函数层. 证明 $H^1(\bS^1, \cR) = 0$.
  \end{enumerate}
\end{exercise}

\begin{proof}
  此处参考 \href{https://math.stackexchange.com/questions/2801221/why-does-the-sheaf-cohomology-of-the-constant-sheaf-on-mathbbr-vanish}{MSE 问题},
  感谢名为 Daniel Schpler 的 MSE 用户.

  我们先处理 $\R$ (配备通常的拓扑) 上的上同调.
  定义 $\R$ 上的层 $\cF$ 是\emph{区间松}的当且仅当对任意两个开区间 $I \supset J$, 都有 $\cF(I) \to \cF(J)$ 满.
  下面我们证明区间松的层都是 $\Gamma(\R, -)$ 零调对象.
  \begin{enumerate}
    \item 若 $0 \to \cF' \to \cF \to \cF'' \to 0$ 是 $\R$ 上层的正合列, 且 $\cF'$ 区间松, 则
          \[
          0 \to \Gamma(U, \cF') \to \Gamma(U, \cF) \to \Gamma(U, \cF'') \to 0
          \]
          对所有开集 $U$ 正合. 由于开集总是开区间的不交并, 只需对 $U$ 为开区间证明.
          只需证明 $\Gamma(\R, \cF) \to \Gamma(\R, \cF'')$ 满.
          若 $t \in \Gamma(\R, \cF'')$, 设 $(I, s)$ 为其极大的区间上的提升.
          在 $I$ 的端点附近取区间 $J$ 及 $s' \in \Gamma(J, \cF)$.
          取 $r \in \Gamma(I, \cF)$ 使得 $r|_{I \cap J}$ 映射到 $s - s'$ (这由区间松得到),
          则 $s - r$ 和 $s'$ 可拼成 $I \cup J$ 上 $t$ 的提升.
    \item 若 $0 \to \cF' \to \cF \to \cF'' \to 0$ 是 $\R$ 上层的正合列, 且 $\cF', \cF$ 区间松,
          则 $\cF''$ 也区间松. 由上一条显然.
    \item 内射层都区间松. 因为内射层都松, 所以也区间松.
  \end{enumerate}
  综上即可证明 $H^i(\R, \cF) = 0$ 对任意 $i$ 和任意区间松 $\cF$ 成立.

  接下来证明原问题.
  \begin{enumerate}[label=(\arabic*)]
    \item 取 $I_u, I_d$ 为 $\bS^1$ 的上下半圆 (闭区间), 其交集为 $\{P, Q\}$.
          记 $\Z_u, \Z_d$ 为 $\Z$ 在 $I_u, I_d$ 上的限制 (并在 $I_u, I_d$ 外用 $0$ 延拓), 则显然有正合列
          \[
          0 \to \Z \xrightarrow{\begin{bmatrix} 1 \\ 1 \end{bmatrix}} \Z_u \oplus \Z_d \xrightarrow{\begin{bmatrix} 1 & 1 \\ -1 & -1 \end{bmatrix}} \Z_P \oplus \Z_Q \to 0.
          \]
          正合性可在茎上逐点验证. 取其上同调长正合列得
          \[
          0 \to \Z \xrightarrow{\begin{bmatrix} 1 \\ 1 \end{bmatrix}} \Z^2 \xrightarrow{\begin{bmatrix} 1 & 1 \\ -1 & -1 \end{bmatrix}} \Z^2
          \to H^1(\bS^1, \Z) \to H^1(I_u, \Z_u) \oplus H^1(I_d, \Z_d) \to 0.
          \]
          而 $I_u$ 可以同构于 $\R$ 的闭子区间 $[0, 1]$. 将 $Z_u$ 以 $0$ 延拓到 $\R$ 上后, 显然其区间松. 因此 $H^1(I_u, \Z_u) = 0$.
          同理, $H^1(I_d, \Z_d) = 0$. 因此即可得出 $H^1(\bS^1, \Z) \cong \coker\bigl(\bigl[\begin{smallmatrix} 1 & 1 \\ -1 & -1 \end{smallmatrix}\bigr]\bigr) \cong \Z$.
    \item 同上取 $\cR_u, \cR_d$, 记 $C(U, \R)$ 为 $U$ 上的连续函数族, 依然有长正合列
          \[
          0 \to C(\bS^1, \R) \to C(I_u, \R) \oplus C(I_d, \R) \to \R^2
          \to H^1(\bS^1, \cR) \to H^1(I_u, \cR_u) \oplus H^1(I_d, \cR_d) \to 0.
          \]
          同理, $\cR_{u, d}$ 以 $0$ 延拓到 $\R$ 上后区间松, 因此上同调消失. 而 $C(I_u, \R) \oplus C(I_d, \R) \to \R^2$ 是满射.
          因此 $H^1(\bS^1, \cR) = 0$. \qedhere
  \end{enumerate}
\end{proof}

\end{document}
