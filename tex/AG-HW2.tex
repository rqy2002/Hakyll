\documentclass{article}
\usepackage{xeCJK}
\usepackage{fontspec}
\usepackage{amsmath}
\usepackage{amsthm}
\usepackage{unicode-math}
\setmathfont{Libertinus Math}
\usepackage[a4paper, margin=2cm]{geometry}
\usepackage{enumitem}
\usepackage{tikz-cd}
\usepackage{hyperref}
\hypersetup{
  colorlinks = true,
  linkcolor = blue
}

\def\proofname{证明}

\newtheoremstyle{exercise}%
{3pt}{3pt}
{}%
{}%
{\bfseries}%
{.}%
{.2em}%
{}
\theoremstyle{exercise}
\newtheorem{exercise}{习题}[section]
\theoremstyle{plain}
\newtheorem*{lemma*}{引理}
\theoremstyle{remark}
\newtheorem*{remark*}{注}
\newenvironment{proofc}{\proof}{\endproof}
\def\printfootnotes{}

%--------- copy from btex ---------%

\def\ga{\mathfrak{a}}
\def\gp{\mathfrak{p}}
\def\gq{\mathfrak{q}}
\def\gm{\mathfrak{m}}
\def\gn{\mathfrak{n}}
\def\A{\mathbb{A}}
\def\P{\mathbb{P}}
\def\Z{\mathbb{Z}}
\def\F{\mathbb{F}}
\def\Q{\mathbb{Q}}
\def\R{\mathbb{R}}
\def\C{\mathbb{C}}
\def\bS{\mathbf{S}}
\def\Ab{\mathfrak{Ab}}
\def\id{\mathrm{id}}
\def\sO{\mathscr{O}}
\def\sK{\mathscr{K}}
\def\sF{\mathscr{F}}
\def\sE{\mathscr{E}}
\def\sG{\mathscr{G}}
\def\sH{\mathscr{H}}
\def\sI{\mathscr{I}}
\def\sP{\mathscr{P}}
\def\sR{\mathscr{R}}
\def\cC{\mathcal{C}}
\def\Sch{\mathfrak{Sch}}
\def\Rings{\mathfrak{Rings}}
\def\spto{\rightsquigarrow}
\def\red{\mathrm{red}}
\def\spp{\operatorname{sp}}
\def\Hom{\operatorname{Hom}}
\def\sHom{\mathop{\mathscr{H\hspace{-3.3pt}o\hspace{-0.8pt}m\hspace{-0.5pt}}}}
\def\Spec{\operatorname{Spec}}
\def\Tot{\operatorname{Tot}}
\def\Proj{\operatorname{Proj}}
\def\Supp{\operatorname{Supp}}
\def\Ext{\operatorname{Ext}}
\def\sExt{\mathop{\mathscr{E\hspace{-2pt}x\hspace{-0.8pt}t}}}
\def\coker{\operatorname{coker}}
\def\im{\operatorname{im}}
\def\dim{\operatorname{dim}}
\def\codim{\operatorname{codim}}
\def\height{\operatorname{height}}
\def\Frac{\operatorname{Frac}}
\def\trd{\operatorname{tr.d.}}
\def\leq{\leqslant}
\def\geq{\geqslant}
\def\Stacks#1{\href{https://stacks.math.columbia.edu/tag/#1}{Stacks Project #1}}
\def\clearfootnotes{\def\@printfootnotes{}}

\begin{document}

代数几何课程的两次习题.

\section{第二次作业}

\begin{exercise}
  设 $X$ 为 Noether 空间.
  \begin{enumerate}[label=(\roman*)]
    \item 设 $\{\sF_i\}$ 为 $X$ 上的 Abel 群层. 则
          \[
          H^n(X, \bigoplus_i \sF_i) \cong \bigoplus_i H^n(X, \sF_i) \quad \forall n.
          \]
    \item 设 $J$ 为有向集, $(\{\sF_i\}, \{\varphi_{i,j}\})_{i \in J}$ 为 $X$ 上 Abel 群层的有向系.
          则
          \[
          H^n(X, \varinjlim_i \sF_i) \cong \varinjlim_i H^n(X, \sF_i) \quad \forall n.
          \]
  \end{enumerate}
\end{exercise}

\begin{proofc}
  在 (ii) 中取 $J$ 为 $I$ 的所有有限子集就知道 (i) 是 (ii) 的直接推论. (有限直和显然保持同调).

  据 Hartshorne 习题 II.1.11, 由于 $X$ 是 Noether 空间, 有
  \[
    \Gamma(X, \varinjlim_i \sF_i) \cong \varinjlim_i \Gamma(X, \sF_i).
  \]
  这同时也说明若诸 $\sF_i$ 都为松层, 则 $\varinjlim_i \sF_i$ 也松; 且 $\varinjlim_i$ 正合.

  因此我们取 $\sF_i$ 的内射消解 $\sI_i^\bullet$, 则 $\varinjlim_i \sI_i^\bullet$ 即为 $\varinjlim_i \sF_i$ 的零调消解.
  取上同调, 由正合性即得结论.
\end{proofc}

\begin{exercise} \hfill
  \begin{enumerate}[label=(\roman*)]
    \item 设 $f \colon X \to Y$ 连续. 则对 $X$ 上任意 Abel 群层 $\sF$ 和任意 $i$,
          $R^i f_* \sF$ 恰为预层
          \[
          V \mapsto H^i(f^{-1}V, \sF)
          \]
          的层化.
    \item 设 $(X, \sO_X)$ 为环化空间, $\sF, \sG$ 为 $\sO_X$ 模. 则对任意 $i$,
          $\sExt^i_{\sO_X}(\sF, \sG)$ 即为预层
          \[
          U \mapsto \Ext^i_{\sO_X|_U}(\sF|_U, \sG|_U)
          \]
          的层化.
  \end{enumerate}
\end{exercise}

\begin{proofc} \hfill
  \begin{enumerate}[label=(\roman*)]
    \item 设 $\sI^\bullet$ 为 $\sF$ 的内射消解. 则 $R^i f_* \sF$ 按定义即为 $f_* \sI^\bullet$ 的上同调.
          而 $f_* \sI^\bullet$ 作为预层的上同调即为 $V \mapsto H^i(f^{-1}V, \sF)$.
          层化函子是预层范畴到层范畴的正合函子, 因此其保持链复形的上同调, 因而
          $f_* \sI^\bullet$ 作为层的上同调即为预层 $V \mapsto H^i(f^{-1}V, \sF)$ 的层化.
    \item 类似的, 我们取 $\sG$ 的内射消解 $\sI$. 则 $\sExt^i_{\sO} = H^i(\sHom_{\sO_X}(\sF, \sI^\bullet))$.
          按定义, $\sHom_{\sO_X}(\sF, \sI^\bullet)$ 即为层 $U \mapsto \Hom_{\sO_X|U}(\sF|_U, \sI^\bullet|_U)$ 构成的链复形.
          因此作为预层, 其上同调即为 $U \mapsto \Ext^i_{\sO_X|_U}(\sF|_U, \sG|_U)$.
          而层化保链复形上同调, 因此此链复形作为层的上同调即为上述预层的层化. \qedhere
  \end{enumerate}
\end{proofc}

\begin{exercise}
  设 $(X, \sO_X)$ 为环化空间, $\sF$ 为 $\sO_X$ 模. 令
  \[
    0 \to \sG' \to \sG \to \sG'' \to 0
  \]
  为 $\sO_X$ 模范畴中的短正合列. 证明有长正合列
  \begin{align*}
    \cdots &\to \Ext^i_{\sO_X}(\sG'', \sF) \to \Ext^i_{\sO_X}(\sG, \sF) \to \Ext^i_{\sO_X}(\sG', \sF) \xrightarrow{\delta} \\
    &\to \Ext^{i+1}_{\sO_X}(\sG'', \sF) \to \cdots.
  \end{align*}
\end{exercise}

\begin{proofc}
  只需证明 $\Ext^i_{\sO_X}(-, \sF)$ 亦是反变函子 $\Hom_{\sO_X}(-, \sF)$ 的右导出函子.
  这是同调代数的基本结果.

  可以通过同时取 $\sG$ 的投射消解 $\sP^\bullet$ 和 $\sF$ 的内射消解 $\sI^\bullet$,
  直和得到双重链复形, 取其两种不同的滤结构求谱序列即得到 $H^i(\Hom_{\sO_X}(\sG, \sI^\bullet)) \cong H^i(\Tot^\oplus(\Hom_{\sO_X}(\sP^\bullet, \sI^\bullet))) \cong H^i(\Hom_{\sO_X}(\sP^\bullet, \sF))$.
\end{proofc}

\begin{exercise}
  设 $(X, \sO_X)$ 为环化空间, $\sF, \sG$ 为 $\sO_X$ 模.
  定义 $\sF$ 对 $\sG$ 的\emph{扩张}为形如
  \[
    0 \to \sG \to \sE \to \sF \to 0
  \]
  的短正合列; 两个扩张 $\sE, \sE'$ 等价当且仅当有交换图
  \[
    \begin{tikzcd}
      0 \ar[r] & \sG \ar[r] \ar[d, "="] & \sE \ar[r] \ar[d, "\cong"] & \sF \ar[r] \ar[d, "="] & 0 \\
      0 \ar[r] & \sG \ar[r]             & \sE' \ar[r]                & \sF \ar[r]             & 0.
    \end{tikzcd}
  \]
  证明存在自然的一一对应
  \[
    \Ext^1_{\sO_X}(\sF, \sG) \xrightarrow{\sim} \{ \sF\ \text{对}\ \sG\ \text{的扩张}\} / \cong.
  \]
\end{exercise}

\begin{proofc}
  这也是同调代数的基本结果. 我们不管 $(X, \sO_X)$, 改为在任意有足够投射对象的 Abel 范畴 $\cC$ 中考虑.

  记 $A, B \in \cC$. 我们记 $e(A, B)$ 为所有 $A$ 对 $B$ 的扩张构成的的等价类的集合.
  记 $\cdots \to P_1 \xrightarrow{d} P_0 \xrightarrow{\pi} A$ 为 $A$ 的投射消解.
  则对任意扩张 $0 \to B \to E \to A \to 0$, 我们有提升
  \[
    \begin{tikzcd}
      \cdots \ar[r] & P_1 \ar[r, "d"] \ar[d] & P_0 \ar[r, "\pi"] \ar[d] & A \ar[r] \ar[d, "="] & 0 \\
      0      \ar[r] & B   \ar[r]             & E   \ar[r]               & A \ar[r]             & 0.
    \end{tikzcd}
  \]
  且这样的提升在同伦意义下唯一. 此映射 $\varphi \colon P_1 \to B$ 是链复形 $\Hom(P_\bullet, B)$ 中的上环,
  因此给出了 $[\varphi] \in H^1(\Hom(P_\bullet, B)) = \Ext^1(A, B)$.
  同伦唯一性说明此上同调类不依赖于提升的选择 (这就是 $\id_B \in \Hom(B, B)$ 对应的上同调类的拉回).

  记 $E'$ 为 $B \xleftarrow{\varphi} P_1 \xrightarrow{d} P_0$. 则按定义有交换图表
  \[
    \begin{tikzcd}
      \cdots \ar[r] & P_1 \ar[r, "d"] \ar[d] & P_0 \ar[r, "\pi"] \ar[d] & A \ar[r] \ar[d, "="] & 0 \\
      0      \ar[r] & B   \ar[r] \ar[d, "="] & E'  \ar[r]        \ar[d] & A \ar[r] \ar[d, "="] & 0 \\
      0      \ar[r] & B   \ar[r]             & E   \ar[r]               & A \ar[r]             & 0.
    \end{tikzcd}
  \]
  其中 $\psi \colon E' \to A$ 由 $B \xrightarrow{0} A \xleftarrow{\pi} P_0$ 诱导.
  第二行事实上也正合: 其显然在 $B$ 和 $A$ 处正合, 且是链复形.
  若 $p \in P_0, b \in B$, 使得 $\psi(p + b) = 0$, 则 $\pi(p) = 0$. 因此存在 $p_1 \in P_1$ 使得 $p_0 = d(p_1)$.
  因而在 $E'$ 中有 $p + b = d(p_0) + b \in B$.

  只关注后两行. 由五引理, $E \cong E'$. 因此扩张 $E \cong E' \in e(A, B)$ 反过来被 $\varphi$ 唯一确定.

  综上, 我们已经有如下交换图表
  \[
    \begin{tikzcd}[sep=4em]
      e(A, B) \ar[dr, "E \mapsto \lbrack\varphi\rbrack"'] & \Hom(P_1, B) \ar[l, "\varphi \mapsto B \amalg P_0"'] \ar[d] \\
       & \Ext^1(A, B).
    \end{tikzcd}
  \]
  只需再证明若 $\varphi, \varphi' \in \Hom(P_1, B)$ 给出同一个等价类, 则他们给出同一个扩张.
  而此时存在 $\psi \in \Hom(P_0, B)$ 使得 $\varphi - \varphi' = \psi d$.
  因此 $E = B \amalg_{P_1, \varphi} P_0$ 和 $E' = B \amalg_{P_1, \varphi'} P_0$ 之间有同构 $(p, b) \mapsto (p, b + \psi(p))$.
  所以上述映射 $\varphi \mapsto B \amalg_{P_1, \varphi} P_0$ 确实诱导了 $\Ext^1(A, B) \to e(A, B)$ 的映射,
  其与 $E \mapsto [\varphi]$ 互逆. 这样就给出了一一对应.
\end{proofc}

\section{第五次作业}

\begin{exercise}
  考虑复形的两项滤过 $K^\bullet = F^0 K^\bullet \supset F^1 K^\bullet \supset F^2 K^\bullet \supset 0$.
  写出其对应的谱序列, 并与复形间短正合列给出的长正合列作比较.
\end{exercise}

\begin{proofc}
  记 $K^{\prime\bullet} = F^1 K^\bullet, K^{\prime\prime\bullet} = K^\bullet / K^{\prime\bullet}$.
  谱序列的 $E_1$ 页为 $E_1^{p, q} = H^{p + q}(\operatorname{gr}^p K^\bullet)$.
  \[
    E_1^{p, q} = \begin{cases}
      H^q(K^{\prime\prime\bullet}) & p = 0, \\
      H^{q + 1} (K^{\prime\bullet}) & p = 1. \\
      0 & p \neq 0, 1
    \end{cases}
  \]
  因此其 $E_2$ 页即退化. 设 $\delta^q \colon H^q(K^{\prime\prime\bullet}) \to H^{q + 1}(K^{\prime\bullet})$, 则
  \[
    E_2^{p, q} = \begin{cases}
      \ker \delta^q & p = 0,\\
      \coker \delta^q & p = 1.
    \end{cases}
  \]
  因此有短正合列 $0 \to \coker \delta^q \to H^{q + 1}(K^\bullet) \to \ker \delta^{q + 1}$.
  
  把这些短正合列拼起来, 就得到
  \[
    \cdots \to H^q(K^{\prime\prime\bullet}) \xrightarrow{\delta} H^{q + 1}(K^{\prime\bullet}) \to H^{q + 1}(K^\bullet)
    \to H^{q + 1}(K^{\prime\prime\bullet}) \xrightarrow{\delta} H^{q + 2}(K^{\prime\bullet}) \to \cdots
  \]
  也就是对应的长正合列.
\end{proofc}

\begin{exercise}
  证明某书上某引理:
  若谱序列 $E_2^{p, q} \Rightarrow H^{p + q}$ 满足只要 $q \neq q_1, q_2$ 就有 $E_2^{p, q} = 0$ ($q_1 < q_2$),
  求证有长正合列
  \[
    \cdots \to E_2^{n - q_1, q_1} \to H^n \to E_2^{n - q_2, q_2} \to E_2^{n + 1 - q_1, q_1} \to H^{n + 1} \to \dots.
  \]
\end{exercise}

\begin{proofc}
  记 $r_0 = q_2 - q_1 + 1$.
  显然对任意 $r = 2, \dots, r_0 - 1$, 都有 $d_r^{p, q} \colon E_r^{p, q} \to E_r^{p + r, q - r + 1}$ 为 $0$.
  因此 $E_{r_0}^{p, q} = E_2^{p, q}$.
  
  又此谱序列显然在 $r_0 + 1$ 页后退化 (之后的 $d = 0$), 则有
  $0 \to E_{r_0 + 1}^{n - q_1, q_1} \to H^n \to E_{r_0 + 1}^{n - q_2, q_2} \to 0$.
  记 $\delta^n = d_{r_0}^{n - q_2, q_1} \colon E_2^{n - q_2, q_2} \to E_2^{n + 1 - q_1, q_1}$,
  则 $E_{r_0 + 1}^{n - q_1, q_1} = \coker \delta^{n-1}, E_{r_0 + 1}^{n - q_2, q_2} = \ker \delta^n$.
  因此
  $0 \to \coker \delta^{n-1} \to H^n \to \delta^n \to 0$.
  将这些短正合列拼起来即得到所需求的长正合列.
\end{proofc}

\end{document}
