\documentclass{article}
\usepackage{xeCJK}
\usepackage{fontspec}
\usepackage{amsmath}
\usepackage{amsthm}
\usepackage{unicode-math}
\setmathfont{Libertinus Math}
\usepackage[a4paper, margin=2cm]{geometry}
\usepackage{enumitem}
\usepackage{tikz-cd}
\usepackage{hyperref}
\hypersetup{
  colorlinks = true,
  linkcolor = blue
}

\def\proofname{证明}

\newtheoremstyle{exercise}%
{3pt}{3pt}
{}%
{}%
{\bfseries}%
{.}%
{.2em}%
{}
\theoremstyle{exercise}
\newtheorem{exercise}{习题}[section]
\newenvironment{proofc}{\proof}{\endproof}
\def\printfootnotes{}

%--------- copy from btex ---------%

\def\ga{\mathfrak{a}}
\def\gp{\mathfrak{p}}
\def\gq{\mathfrak{q}}
\def\gm{\mathfrak{m}}
\def\gn{\mathfrak{n}}
\def\A{\mathbf{A}}
\def\Z{\mathbb{Z}}
\def\F{\mathbb{F}}
\def\Q{\mathbb{Q}}
\def\R{\mathbb{R}}
\def\C{\mathbb{C}}
\def\id{\mathrm{id}}
\def\sO{\mathcal{O}}
\def\Sch{\mathfrak{Sch}}
\def\Rings{\mathfrak{Rings}}
\def\spto{\rightsquigarrow}
\def\red{\mathrm{red}}
\def\spp{\operatorname{sp}}
\def\Hom{\operatorname{Hom}}
\def\Spec{\operatorname{Spec}}
\def\Proj{\operatorname{Proj}}
\def\Supp{\operatorname{Supp}}
\def\coker{\operatorname{coker}}
\def\im{\operatorname{im}}
\def\dim{\operatorname{dim}}
\def\codim{\operatorname{codim}}
\def\height{\operatorname{height}}
\def\Frac{\operatorname{Frac}}
\def\trd{\operatorname{tr.d.}}
\def\leq{\leqslant}
\def\geq{\geqslant}
\def\Stacks#1{\href{https://stacks.math.columbia.edu/tag/#1}{Stacks Project #1}}
\def\clearfootnotes{\def\@printfootnotes{}}

\begin{document}

这里是 Hartshorne 第二章第三节习题.

\setcounter{section}{2}

\section{概形的基本性质}

\begin{exercise}
  证明概形同态 $f \colon X \to Y$ 局部有限型当且仅当对 $Y$ 中\emph{任意}仿射开集 $V = \Spec B$,
  $f^{-1}(V)$ 都可以由若干仿射开集 $U_j = \Spec A_j$ 覆盖, 其中 $A_j$ 都是有限生成 $B$-代数.
\end{exercise}

\begin{proofc}
  充分性显然, 只需证明必要性.
  首先要证明:
  若 $Y$ 中仿射开集 $V = \Spec B$ 满足条件, $b \in B$,
  则 $D(b) = \Spec B_b \subseteq V$ 也满足条件.
  这是因为对每个 $U_j = \Spec A_j$, 都有 $U_j \cap f^{-1}(D(b)) = \Spec (A_j)_{\bar{b}}$,
  且 $(A_j)_{\bar{b}}$ 是有限生成 $B_b$-代数 ($\bar{b}$ 是 $b$ 的像).
  因此满足条件的仿射开集 $V \subseteq Y$ 构成 $Y$ 的一组基.

  再设 $V = \Spec B$ 是 $Y$ 中任意的仿射开集.
  则 $V$ 可以由有限个基本开集 $D(b_i)$ 覆盖, 且每个 $D(b_i)$ 都满足条件,
  即存在若干 $U_{ij} = \Spec A_{ij}$ 覆盖 $f^{-1}(D(b_i))$,
  使得 $A_{ij}$ 是有限生成 $B_{b_i}$ 代数.
  因此它们也都是有限生成 $B$ 代数, 并且覆盖 $f^{-1}(V)$. 因此 $V$ 也满足条件.
\end{proofc}

\begin{exercise}
  设 $f \colon X \to Y$ 是概形同态. 若 $Y$ 可以由若干仿射开集 $V_i$ 覆盖,
  且其中每个 $f^{-1}(V_i)$ 都拟紧, 就称 $f$\emph{拟紧}.
  证明 $f$ 拟紧当且仅当对\emph{任意}仿射开集 $V \subset Y$, $f^{-1} (V)$ 都拟紧.
\end{exercise}

\begin{proofc}
  充分性显然, 只需证明必要性.
  显然, 概形中的开集拟紧当且仅当其可以被有限个仿射开集覆盖.

  设 $V = \Spec B$ 满足 $f^{-1}(V)$ 拟紧, 记其被 $U_1, \dots, U_k$ 覆盖,
  其中 $U_i = \Spec A_i$. 则对任意 $b \in B$,
  $f^{-1}(D(b))$ 可以被 $U_i \cap f^{-1}(D(b)) = \Spec (A_i)_{\bar{b}}$ 覆盖.
  因此满足条件的开集构成基.

  再设 $V = \Spec B$ 是 $Y$ 中任意仿射开集, 则其可以有有限个基本开集 $D(b_i)$ 覆盖,
  其中每个 $f^{-1}(D(b_i))$ 拟紧. 因此 $f^{-1}(V) = \bigcup f^{-1}(D(b_i))$ 拟紧.
\end{proofc}

\begin{exercise}
  \begin{enumerate}[label={(\alph*)}]
    \item 证明概形同态 $f \colon X \to Y$ 有限型当且仅当其局部有限型且拟紧.
    \item 由此说明 $f$ 有限型当且仅当对 $Y$ 中\emph{任意}仿射开集 $V = \Spec B$,
          $f^{-1}(V)$ 都可以被有限个仿射开集 $U_j = \Spec A_j$ 覆盖,
          其中每个 $A_j$ 都是有限生成 $B$-代数.
    \item 证明如果 $f$ 有限型, 则对 $Y$ 中\emph{任意}仿射开集 $V = \Spec B$,
          以及 $X$ 中任意仿射开集 $U = \Spec A \subseteq f^{-1}(V)$,
          $A$ 都是有限生成 $B$-代数.
  \end{enumerate}
\end{exercise}

\begin{proofc}
  \begin{enumerate}[label={(\alph*)}]
    \item 若 $f$ 有限型, 则其当然局部有限型.
          且若 $Y$ 中的仿射开集 $V = \Spec B$ 使得 $f^{-1}(V)$ 可以被有限个仿射开集覆盖,
          则 $f^{-1}(V)$ 当然拟紧. 因此 $f$ 拟紧.

          反之, 若 $f$ 局部有限型且拟紧, 则由前两习题知:
          对 $Y$ 中任意仿射开集 $V = \Spec B$,
          $f^{-1}(V)$ 可以被若干仿射开集 $U_i = \Spec A_i$ 覆盖,
          每个 $A_i$ 都是有限生成 $B$-代数. 而 $f^{-1}(V)$ 又拟紧, 从而可以被这其中有限个所覆盖.
          因此 $f$ 有限型.
    \item 由前两习题及 (a) 即证.
    \item 固定 $V = \Spec B$. 若 $U = \Spec A \subset f^{-1}(V)$ 满足 $A$ 是有限生成 $B$-代数,
          则对任意 $a \in A$, $A_a$ 也是有限生成 $B$-代数.
          因此 $f^{-1}(V)$ 中所有满足 $U = \Spec A$ 且 $A$ 是有限生成 $B$-代数的仿射开集构成一组基.

          现在任取 $f^{-1}(V)$ 中的仿射开集 $U = \Spec A$.
          则存在有限个 $a_i \in A$, 它们生成 $A$, 且每个 $A_{a_i}$ 都是有限生成 $B$-代数.
          设 $n$ 是足够大的正整数, 使得每个 $A_{a_i}$ 都可以通过 $a_i^{-n} x_{ij}$ 在 $B$ 上生成.
          由于所有 $a_i$ 生成 $A$, 所有 $a_i^n$ 也生成 $A$.
          不妨设 $1 = \sum_i y_i a_i^n$. 则对任意 $a \in A$, $a = \sum_i y_i (a_i^n a)$.
          因此易知 $A$ 可以由 $\{ x_{ij} \} \cup \{ y_i \}$ 在 $B$ 上生成, 从而是有限生成 $B$-代数.
          \qedhere
  \end{enumerate}
\end{proofc}

\begin{exercise}
  证明:概形同态 $f \colon X \to Y$ 有限当且仅当对 $Y$ 中\emph{任意}仿射开集 $V = \Spec B$,
  $f^{-1}(V)$ 都是仿射开集, 且若记 $f^{-1}(V) = \Spec A$, 则 $A$ 在 $B$ 上有限.
\end{exercise}

\begin{proofc}
  充分性显然.

  设 $f$ 有限, $V = \Spec B \subseteq Y, U = f^{-1}(V) = \Spec A$,
  且 $A$ 是有限 $B$-模. 记 $A \to B$ 的环同态是 $\varphi$,
  则对任意 $b \in B$, $f^{-1}(D(b)) = \Spec A_{\varphi(b)}$ 是有限 $B_b$-模.
  再设 $V = \Spec B \subseteq Y$ 是任意仿射开集, $U = f^{-1}(V)$.
  则由上可知存在有限个 $b_i \in B$, 它们生成 $B$,
  且每个 $f^{-1}(D(b_i)) = \Spec A_i$, 对应的 $A_i$ 是有限 $B_{b_i}$-模.
  设 $a_i = f^\sharp(V)(b_i) \in \sO_X(U)$.
  则 $a_i$ 也生成 $\sO_X(U)$, 且每个 $U_{a_i} = f^{-1}(D(b_i))$ 仿射.
  因此由习题 2.17 即知 $U$ 也仿射. 记 $U = \Spec A$.
  则每个 $A_{a_i}$ 是有限 $B_{b_i}$ 模.
  接下来类似前一习题中的 (c) 易证 $A$ 是有限 $B$ 模.
\end{proofc}

\begin{exercise}
  设 $f \colon X \to Y$ 是概形态射. 若对每个 $y \in Y$, $f^{-1}(y)$ 都是有限集,
  就称 $f$ \emph{拟有限}.
  \begin{enumerate}[label={(\alph*)}]
    \item 证明有限态射也拟有限.
    \item 证明有限态射是\emph{闭映射}, 即其将任意闭子集映射到闭子集.
    \item 给出反例以证明有限型、拟有限、闭的满概形态射不一定是有限态射.
  \end{enumerate}
\end{exercise}

\begin{proofc}
  \begin{enumerate}[label={(\alph*)}]
    \item 若 $y \in Y$, 取包含 $y$ 的仿射开集 $V = \Spec B$.
          记 $f^{-1}(V) = \Spec A$, $y$ 对应 $B$ 中的素理想 $\gp$.
          则 $f^{-1}(y)$ (至少作为拓扑空间) 同胚于 $\Spec (B \otimes_A k(\gp))$,
          其中 $k(\gp) = A_{\gp} / \gp$ 是 $\gp$ 的剩余域.

          而 $B \otimes_A k(\gp)$ 作为模是有限维 $k(\gp)$-线性空间, 从而只包含有限个素理想.
    \item 在任意仿射开集上, 这就是上行性质. 由于概形被仿射开集覆盖, 命题即证. % TODO: 译名
    \item 取“两个原点的直线”到直线的映射即可. 显然其有限型, 拟有限, 满且闭. % TODO: 没验证.
          但双原点的直线并不仿射, 因此这个映射并不有限.

          又或者令 $X = \Spec \Z[i]_{(1+2i)}$, 其中 $i^2 = -1$.
          则 $X \to \Spec \Z$ 有限型, 拟有限, 满且闭. 然而其不有限.
          \qedhere
  \end{enumerate}
\end{proofc}

\begin{exercise}
  设 $X$ 是整概形. 证明一般点 $\xi$ 处的局部环 $\sO_{\xi}$ 是域.
  其被称作 $X$ 的\emph{函数域}, 记作 $K(X)$.
  证明如果 $U = \Spec A$ 是 $X$ 的任意仿射开集, 则 $K(X)$ 同构于 $A$ 的分式域.
\end{exercise}

\begin{proofc}
  设 $U = \Spec A$ 是任意仿射开集, 则 $\xi \in U$, 且 $\xi$ 也是 $U$ 的一般点.
  因此 $\xi$ 对应 $A$ 中的零理想, 从而 $\sO_{\xi}$ 同构于 $A$ 的分式域.
\end{proofc}

\begin{exercise}
  设 $f \colon X \to Y$ 是概形同态, $Y$ 不可约,
  如果对 $Y$ 的一般点 $\eta$, $f^{-1}(\eta)$ 是有限集, 就称 $f$\emph{一般有限}.
  如果概形同态 $f \colon X \to Y$ 的像集 $f(X)$ 在 $Y$ 中稠密, 就称 $f$ \emph{支配}.
  现设 $f \colon X \to Y$ 是整概形之间的支配、一般有限、有限型同态.
  证明存在稠密开集 $U \subseteq Y$ 使得诱导映射 $f^{-1}(U) \to U$ 有限
  [\emph{提示:先证明 $X$ 的函数域是 $Y$ 的函数域的有限扩张}].
\end{exercise}

\begin{proofc}
  设 $\xi, \eta$ 分别是 $X, Y$ 的一般点, $K, L$ 分别为 $X, Y$ 的函数域.
  任取 $Y$ 中的仿射开集 $V = \Spec A$,
  以及 $f^{-1}(V)$ 中的仿射开集 $U = \Spec A \subseteq f^{-1}(V)$.
  由于 $f$ 支配且 $X$ 不可约, $f(U)$ 在 $V$ 中稠密,
  因此 $A \otimes_B L$ 非零.
  又因为 $f$ 有限型、一般有限, $A \otimes_B L$ 在 $L$ 上有限生成,
  且仅包含有限个素理想 (与 $f^{-1}(\eta)$ 一一对应).

  由 Noether 正规化定理, 存在 $A \otimes_B L$ 的子环 $C \cong L[y_1, \dots, y_k]$ 使得
  $A \otimes_B L$ 在 $C$ 上整. 那么由上述一般有限性即知 $k = 0$,
  从而 $A \otimes_B L$ 在 $L$ 上整 (因此有限), 所以其分式域 $K$ 是 $L$ 的有限扩张.

  更进一步地, 设 $f^{-1}(V)$ 可以由有限个仿射开集 $U_i = \Spec A_i$ 覆盖,
  设 $A_i \otimes_B L$ 作为 $L$-代数可以由 $x_{ij} \in A_i$ 生成.
  由于它们在 $L$ 上整, 即满足 $L$ 系数首一多项式.
  令 $f \in B$ 为这些多项式系数的分母的乘积, 则 $x_{ij}$ 在 $B_f$ 上整;
  因此 $(A_i)_f$ 在 $B_f$ 上整.

  用 $D(f) = \Spec B_f$ 代替 $Y$, 用 $f^{-1}(D(f))$ 代替 $X$,
  则问题归约为:
  若 $f \colon X \to Y$ 是整概形同态, $Y = \Spec B$,
  且 $X$ 可以被有限个仿射开集 $U_i = \Spec A_i$, 其中每个 $A_i$ 都是有限 $B$-模,
  就存在稠密开集 $V \subseteq Y$ 使得 $f^{-1}(V) \to V$ 有限.

  记 $W = \bigcap_i U_i$. 对每个 $i$, 设 $U_i - W = V(\ga_i), \ga_i \subseteq A_i$.
  由于 $A_i$ 在 $B$ 上整, 存在 $b_i \in B \cap \ga_i$.
  设 $V = \bigcap_i D(b_i) \subseteq Y$, 显然 $V \cong \Spec B[\{ b_i^{-1} \}_i]$.
  且由于 $f^{-1}(V) \subseteq W$,
  即知 $f^{-1}(V) \cong \bigcap_i D(b_i) \cap U_j \cong \Spec A_j[\{ b_i^{-1} \}_i]$
  仿射, 且 $f \colon f^{-1}(V) \to V$ 有限.
  由于 $Y$ 不可约, $V$ 是开集, 从而稠密.
\end{proofc}

\begin{exercise}[正规化]
  若一概形的所有局部环都整闭, 就称其\emph{正规}.
  令 $X$ 为整概形. 对每个仿射开集 $U = \Spec A$, 设 $\tilde{A}$ 是 $A$ 在其分式域中的整闭包,
  令 $\tilde{U} = \Spec \tilde{A}$. 证明这些 $\tilde{U}$ 可以粘接成一个正规概形 $\tilde{X}$,
  称为 $X$ 的\emph{正规化}.
  再证明存在同态 $\tilde{X} \to X$ 满足如下泛性质:
  对任意正规概形 $Z$ 和同态 $f \colon Z \to X$, $f$ 都唯一地穿过 $\tilde{X}$.
  若 $X$ 在域 $k$ 上有限型, 则同态 $\tilde{X} \to X$ 有限.
  这推广了第一章习题 3.17.
\end{exercise}

\begin{proofc}
  仿照构造纤维积的办法, 可以如下证明:

  第一步, 若 $X = \Spec A$, 则 $\tilde{X} = \Spec \tilde{A}$
  配备自然的同态 $\tilde{X} \to X$ 必定满足上述泛性质.
  这可以由习题 2.4 自然得到.

  第二步, 若 $g \colon \tilde{X} \to X$ 满足泛性质,
  $U$ 是 $X$ 的开子概形, 则 $g$ 在 $g^{-1}(U)$ 上的限制 $g^{-1}(U) \to U$ 也满足泛性质.
  若 $Z$ 正规, $f \colon Z \to U$, 则复合嵌入映射得到 $i \circ f \colon Z \to X$.
  由泛性质, $i \colon f$ 唯一地穿过 $\tilde{X}$;显然其像集必定包含在 $g^{-1}(U)$ 中.

  第三步, 若 $U, V$ 都是 $X$ 中的仿射开集, 则 $U \cap V$ 在 $\tilde{U}$ 和 $\tilde{V}$
  中的原像都具有上述泛性质, 因此可以自然地等同.
  这样就给出了将所有 $\tilde{U}$ 粘接为 $\tilde{X}$ 的办法.
  且每个同态 $\tilde{U} \to U \to X$ 也可粘接成 $\tilde{X} \to X$,
  不难验证其满足泛性质.

  接下来设 $X$ 在 $k$ 上有限型. 则 $X$ 可以由有限个仿射开集 $U_i = \Spec A_i$ 覆盖,
  且每个 $A_i$ 都是有限生成 $k$-代数;因此 $\tilde{A}_i$ 在 $A_i$ 上有限.
  所以 $\tilde{X} \to X$ 有限.
\end{proofc}

\begin{exercise}
  回忆在代数簇范畴中, 两个代数簇的乘积的 Zariski 拓扑并不是乘积拓扑 (第一章习题 1.4).
  我们将会看到, 在概形范畴中, 乘积概形的底集合甚至都不是乘积集合.
  \begin{enumerate}[label={(\alph*)}]
    \item 令 $k$ 是域, $\A_k^1 = \Spec k[x]$ 是 $k$ 上的仿射直线.
          证明 $\A_k^1 \times_{\Spec k} \A_k^1 \cong \A_k^2$,
          并证明其底集合并不是两个因子的底集合的乘积 (即使 $k$ 代数闭也一样).
    \item 令 $k$ 是域, $s, t$ 是不定元, 则 $\Spec k, \Spec k(s), \Spec k(t)$
          都是单点空间.

          描述 $\Spec k(s) \times_{\Spec k} \Spec k(t)$.
  \end{enumerate}
\end{exercise}

\begin{proofc}
  \begin{enumerate}[label={(\alph*)}]
    \item 由定义, $\A_k^1 \times_{\Spec k} \A_k^1 = \Spec (k[x] \otimes_k k[x]) = \Spec k[x, y] = \A_k^2$.
          其中任意显含两个变量的不可约多项式生成的素理想
          (如 $(x - y)$) 都不属于两个因子底集合的乘积.
    \item $k, k(s), k(t)$ 都是域, 因此其谱当然是单点空间.

          然而 $\Spec k(s) \times_{\Spec k} \Spec k(t) = \Spec (k(s) \otimes_k k(t))$.
          而 $k(s) \otimes_k k(t) = S^{-1} k[s, t]$,
          其中 $S = \{ f(s) g(t) \mid f, g \in k[x] \setminus \{ 0 \} \}$.
          其素理想为 $(0)$ 以及 $(h(s, t))$, 其中 $h$ 是同时显含 $s$ 和 $t$ 的不可约多项式.
          (hmm... 去掉所有闭点以及平行于坐标轴直线的平面?)
          \qedhere
  \end{enumerate}
\end{proofc}

\begin{exercise}[同态的纤维]
  \begin{enumerate}[label={(\alph*)}]
    \item 若 $f \colon X \to Y$ 是同态, $y \in Y$,
          证明 $\spp(X_y)$ 同胚于装备子空间拓扑的 $f^{-1}(y)$.
    \item 令 $X = \Spec k[s, t] / (s - t^2)$, $Y = \Spec k[s]$,
          $f \colon X \to Y$ 是由 $s \mapsto s$ 决定的同态.
          若 $y \in Y$ 是 $a \in k$ 对应的点且 $a \neq 0$,
          证明纤维 $X_y$ 恰好包含两个点, 剩余域都是 $k$.
          若 $y \in Y$ 对应 $0 \in k$, 则 $X_y$ 是非既约的单点概形.
          若 $\eta \in Y$ 是一般点, 则 $X_{\eta}$ 是单点概形,
          其剩余域是 $\eta$ 处剩余域的二次扩张 (假设 $k$ 代数闭).
  \end{enumerate}
\end{exercise}

\begin{proofc}
  \begin{enumerate}[label={(\alph*)}]
    \item 记同态 $g \colon X_y = X \otimes_Y \Spec k(y) \to X$,
          任取 $Y$ 中包含 $y$ 的仿射开集 $V = \Spec B$.
          只需证明:对 $f^{-1}(V)$ 中任意仿射开集 $U$, $g \colon g^{-1}(U) \to U$
          在底空间上诱导了 $g^{-1}(U)$ 和 $U \cap f^{-1}(y)$ 的同胚.
          而这种情况下, 设 $U = \Spec A$, $y$ 对应 $B$ 中的素理想 $\gp$,
          则
          \[
          g^{-1}(U) = U \otimes_V \Spec k(y) = \Spec (A \otimes_B k(\gp))
          = \Spec (A_{\gp} / \gp A_{\gp}).
          \]
          因此显然.
    \item 若 $y$ 对应 $a \neq 0 \in k$, 则
          \[
          X_y = \Spec \Bigl(k[s, t] / (s - t^2)) \otimes_{k[s]} (k[s] / (s - a)\Bigr)
          = \Spec (k[t] / (t^2 - a)).
          \]
          其包含 $\bigl(t \pm \sqrt{a}\bigr)$ 两个素理想.

          若 $y$ 对应 $0 \in k$, 则同理, $X_y = \Spec (k[t] / (t^2))$
          是非既约的单点概形.

          若 $\eta$ 是一般点, 则 $X_{\eta} = \Spec (k[s, t] / (s - t^2))_s = \Spec k(\sqrt{s})$,
          而 $k(\sqrt{s})$ 是 $\eta$ 处的剩余域 $k(s)$ 的二次扩张.
          \qedhere
  \end{enumerate}
\end{proofc}

\begin{exercise}[闭子概形]
  \begin{enumerate}[label={(\alph*)}]
    \item 闭浸入在基扩张下不变:即若 $f \colon Y \to X$ 是闭浸入, $X' \to X$ 是任意同态,
          则 $f' \colon Y \times_X X' \to X'$ 也是闭浸入.
    \item 若 $Y$ 是仿射概形 $X = \Spec A$ 的闭子概形, 则 $Y$ 仿射;
          事实上 $Y$ 一定是某个闭浸入 $\Spec A / \ga \to \Spec A$ 的像, $\ga$ 是合适的理想.
          [\emph{提示:先证明 $Y$ 可以被有限个形如 $D(f_i) \cap Y$ 的仿射开集覆盖,
          其中 $f_i \in A$. 通过添加一些 $D(f_i) \cap Y = \emptyset$ 的 $f_i$,
          可以假设这些 $D(f_i)$ 覆盖 $X$. 接下来证明 $f_i$ 生成 $A$,
          因此由习题 2.17b 证明 $Y$ 仿射,
          然后用习题 2.18d 证明 $Y$ 可以由某个理想 $\ga \subseteq A$ 得来.}]
    \item 令 $Y$ 是 $X$ 的闭子集, 并为其装备既约诱导闭子集概形结构.
          若 $Y'$ 是 $X$ 中此闭子集上的另一个闭子概形,
          证明闭浸入 $Y \to X$ 穿过 $Y'$.
          我们可以将此性质表达为: 既约诱导闭子概形结构是闭子集上最小的闭子概形结构.
    \item 令 $f \colon Z \to X$ 是概形同态. 则 $X$ 中存在唯一的闭子概形 $Y$ 使得:
          $f$ 穿过 $Y$; 且若 $f$ 也穿过另一个闭子概形 $Y'$, 则 $Y \to X$ 也穿过 $Y'$.
          我们将 $Y$ 称为 $Z$ 的\emph{概形论像}.
          若 $Z$ 既约, 证明 $Y$ 就是 $f(Z)$ 的闭包上的既约诱导闭子概形.
  \end{enumerate}
\end{exercise}

\begin{proofc}
  我们先证明 (b).
  \begin{enumerate}
    \item[(b)] 设 $Y$ 可以由仿射开集 $V_i = \Spec B_i$ 覆盖,
          而 $f(V_i) = \bigl(\bigcup_j U_{ij}\bigr) \cap f(Y)$,
          其中 $U_{ij}$ 是基本开集 $D(f_{ij})$.
          则对每个 $i, j$, $f^{-1}(U_{ij}) = \Spec (B_i)_{f_{ij}} \subseteq V_i$ 也仿射.
          并且由于 $f(Y)$ 是 $X$ 中的闭集,因此也拟紧,从而可以选出有限个 $U_{ij}$ 覆盖 $f(Y)$,
          设为 $\{ D(g_i) \}_i$.

          通过添加一些 $D(g_i) \cap Y = \emptyset$ 的 $g_i$,
          不妨设这些 $D(g_i)$ 覆盖了 $X$;即 $g_i$ 生成了 $A$.
          设 $\varphi \colon A \to \Gamma(Y, \sO_Y)$ 是 $f$ 诱导的整体截面上的映射,
          则 $\varphi(g_i)$ 生成了 $\Gamma(Y, \sO_Y)$.
          且 $Y_{\varphi(g_i)} = f^{-1}(D(g_i))$ 均仿射.
          因此由习题 2.17b 即知 $Y$ 是仿射概形.
          再由习题 2.18d 即知 $Y \cong \Spec A / \ga$.

    \item[(a)] 若 $X, Y, X'$ 都是仿射概形, 设为 $X = \Spec A, Y = \Spec A / \ga, X' = \Spec B$
          (由习题 2.18d, $Y$ 一定形如 $\Spec A / \ga$).
          则 $Y \times_X X' = \Spec B / (B \ga)$ 到 $X'$ 是闭浸入.

          进一步地, 设 $X, Y$ 仍然仿射, $X'$ 为任意概形.
          设 $X'$ 可以由若干仿射开集 $U_i = \Spec B_i$ 覆盖.
          则对每个 $U_i$, $f^{\prime {-1}}(U_i) \cong Y \times_X U_i$ 到 $U_i$ 是闭浸入.
          因此 $Y \times_X X'$ 在 $X'$ 中的像是闭集.
          且由于层上的映射 $f^{\prime\sharp}$ 在每个开集上都是满射, 因此整体上也是满射.
          因此 $Y \times_X X' \to X'$ 也是闭浸入.

          若 $X, Y$ 未必仿射, 则设 $X$ 可以由仿射开集 $X_i = \Spec A_i$ 覆盖.
          设 $X_i$ 在 $Y, X'$ 中的原像分别是 $Y_i, X'_i$.
          显然, $Y_i \to X'_i$ 也仍然是闭浸入, 因此由 (b) 即知 $Y_i$ 仿射.
          由上述论证, 每个 $Y_i \times_{X_i} X'_i \to X'_i$ 都是闭浸入,
          即 $Y \times_X X'_i \to X'_i$ 是闭浸入.
          类似于上述推理即知 $Y \times_X X' \to X'$ 也是闭浸入.
    \item[(c)] 由于可以将映射做粘接, 只需考虑 $X = \Spec A$ 仿射的情况.
          设 $Y = \Spec \ga$. 由 (b) 可知 $Y'$ 也仿射, 设为 $\Spec A / \ga'$,
          则 $\ga = \sqrt{\ga'} \supseteq \ga'$, 因此 $Y \to X$ 穿过 $Y'$.
    \item[(d)] 若 $X = \Spec A$ 仿射, 则可以定义 $\ga = \ker (A \to \Gamma(Z, \sO_Z))$,
          并定义 $Y = \Spec A / \ga$, 显然其满足泛性质.
          此时, $Y$ 在底空间上就是 $f(Z)$ 的闭包.
          且若 $Z$ 既约, 则 $A / \ga$ 也既约, 因此此时 $Y$ 就是既约诱导闭子概形.
          若 $X$ 任意, 则对每个仿射开集定义 $Y$ 之后粘接起来即可.
          \qedhere
  \end{enumerate}
\end{proofc}

\begin{exercise}[$\Proj S$ 的闭子概形]
  \begin{enumerate}[label={(\alph*)}]
    \item 设 $\varphi \colon S \to T$ 是分次环之间保持次数的满射.
          证明习题 2.14 中的开集 $U$ 就等于 $\Proj T$,
          且同态 $\Proj T \to \Proj S$ 是闭浸入.
    \item 若 $I \subseteq S$ 是齐次理想, $T = S / I$,
          令 $Y$ 为由 $\Proj S / I \to \Proj S$ 定义的 $X = \Proj X$ 的闭子概形.
          证明不同的齐次理想可以给出相同的闭子概形.
          例如说, 设 $d_0$ 为整数, $I' = \bigoplus_{d \geq d_0} I_d$,
          则 $I$ 和 $I'$ 决定相同的闭子概形.

          我们之后将会看到 $X$ 的任意闭子概形 (至少在 $S$ 是 $S_0$ 上的多项式环的时候)
          都可以从 $S$ 某个齐次理想得来.
  \end{enumerate}
\end{exercise}

\begin{proofc}
  \begin{enumerate}[label={(\alph*)}]
    \item 回忆 $U = \{ \gp \in \Proj T \mid \gp \not\supseteq \varphi(S_+) \}$.
          若 $S \to T$ 是满射, 则 $S_+ \to T_+$ 也是满射. 因此 $U$ 必然是全空间 $\Proj T$.

          记同态 $f \colon \Proj T \to \Proj S$,
          则 $f^\sharp$ 在茎上的映射是 $S_{(\varphi^{-1}(\gp))} \to T_{(\gp)}$ 都是满射.
          因此 $f^\sharp$ 是满射.
          而 $f(\Proj T) = V(\ker \varphi)$ 是 $\Proj S$ 中的闭集.
          因此 $f$ 是闭浸入.
    \item 若 $I' = \bigoplus_{d \geq d_0} I_d$,
          则 $S / I \to S / I'$ 的映射在不小于 $d_0$ 的次数上都是同构.
          因此由习题 2.14c 即知 $\Proj S / I$ 和 $\Proj S / I'$ 同构.
          \qedhere
  \end{enumerate}
\end{proofc}

\begin{exercise}[有限型同态的性质]
  \begin{enumerate}[label={(\alph*)}]
    \item 闭浸入有限型.
    \item 拟紧的开浸入 (习题 3.2) 有限型.
    \item 有限型同态的复合有限型.
    \item 有限型同态的基扩张仍然有限型.
    \item 若 $X, Y$ 都在 $S$ 上有限型, 则 $X \times_S Y$ 也在 $S$ 上有限型.
    \item 若 $X \xrightarrow{f} Y \xrightarrow{g} Z$ 是概形同态,
          $f$ 拟紧, $g \circ f$ 有限型, 则 $f$ 也有限型.
    \item 若 $f \colon X \to Y$ 有限型, $Y$ Noether, 则 $X$ 也 Noether.
  \end{enumerate}
\end{exercise}

\begin{proofc}
  \begin{enumerate}[label={(\alph*)}]
    \item 若 $f \colon X \to Y$ 是闭浸入,
          由习题 3.11 即知对 $Y$ 中任意仿射开集 $V = \Spec A$,
          其原像都形如 $\Spec A / \ga$. 因此 $f$ 有限型 (甚至有限).
    \item 若 $f \colon X \to Y$ 是开浸入,
          则对 $Y$ 中任意仿射开集 $U$, $f^{-1}(U) \cong f(X) \cap U$
          可以由 $U$ 的若干个仿射开集覆盖, 因此局部有限型.
          若 $f$ 拟紧, 则其有限型.
    \item 显然.
    \item 设 $f \colon X \to S$ 有限型, $g \colon S' \to S$.
          若 $S = \Spec A$ 仿射, 则对 $S'$ 中任意仿射开集 $U = \Spec A'$,
          其在 $X \times_S S'$ 中的原像是 $X \times_S U$.
          因此若 $X$ 可以由有限个仿射开集 $V_i = \Spec B_i$ 覆盖,
          每个 $B_i$ 都是有限生成 $A$-代数,
          则对应地, $X \times_S U$ 也可以由 $\Spec (B_i \otimes_A A')$ 覆盖,
          且 $B_i \otimes_A A'$ 是有限生成 $A'$-代数.
          因此 $f$ 有限型.

          一般情况下, 设 $S$ 可以由若干个仿射开集 $U_i = \Spec A_i$ 覆盖.
          记 $X_i, U_i'$ 为 $U_i$ 在 $X, S'$ 中的原像,
          则 $X \times_S S'$ 可以由 $X_i \times_{U_i} U_i'$ 粘接而成.
          而每个 $X_i \times_{U_i} U_i'$ 在 $U_i'$ 上有限型,
          因此 $X \times_S S'$ 在 $S'$ 上有限型.
    \item 若 $U = \Spec A$ 是 $S$ 中的仿射开集.
          设其在 $X, Y$ 中的原像分别是 $X_0, Y_0$,
          则其在 $X \times_S Y$ 中的原像就是 $X_0 \times_U Y_0$.
          因此若 $X_0, Y_0$ 分别有若干个在 $A$ 上有限型的环对应的仿射开集覆盖,
          则 $X_0 \times_U Y_0$ 就由这些环的张量积对应的仿射开集覆盖,
          从而 $X \times_S Y$ 在 $S$ 上有限型.
    \item 对 $Z$ 中任意仿射开集 $U = \Spec A$,
          若 $V = \Spec B \subseteq g^{-1}(U), W = \Spec A = \subseteq f^{-1}(V)$
          分别是 $Y, X$ 中的仿射开集,
          则由 $g \circ f$ 有限型即知 $C$ 在 $A$ 上有限生成.
          而 $A \to C$ 穿过 $B$, 因此 $C$ 在 $B$ 上有限生成.
          而 $Y$ 中满足这样条件的 $V$ 可以覆盖 $Y$, 因此 $f$ 局部有限型.
          $f$ 又拟紧, 从而有限型.
    \item 若 $Y$ 可以由有限个仿射开集 $V_i = \Spec A_i$ 覆盖,
          其中每个 $A_i$ Noether, 则由于 $f$ 有限型,
          每个 $f^{-1}(V_i)$ 又可以由有限个仿射开集 $U_{ij} = \Spec B_{ij}$ 覆盖,
          其中 $B_{ij}$ 在 $A_i$ 上有限生成. 由 Hilbert 基定理, $B_{ij}$ Noether.
          因此 $X$ Noether.
          \qedhere
  \end{enumerate}
\end{proofc}

\begin{exercise}
  若 $X$ 是域上的有限型概形, 证明 $X$ 的闭点稠密.
  给出反例说明这个结论对一般的概形并不成立.
\end{exercise}

\begin{proofc}
  我们断言: 若 $X$ 是域 $k$ 上的有限型概形,
  则点 $p \in X$ 是闭点当且仅当其剩余域是 $k$ 的有限扩张.

  事实上, 若 $k(p)$ 是 $k$ 的有限扩张,
  则对 $X$ 中任意包含 $p$ 的仿射开集 $U = \Spec A$,
  若 $p$ 对应 $\gp \subseteq A$,
  则 $A / \gp$ 的分式域同构于 $k(p)$.
  但 $A / \gp \to k(p)$ 又是 $k$-同态, 因此 $A / \gp$ 在 $k$ 上有限,
  从而必定是域. 也就是说, $p$ 在 $X$ 的任意仿射子集中闭, 从而在 $X$ 中闭.

  反之, 若 $p \in X$ 是闭点, 则任取包含 $p$ 的仿射开集 $\Spec A$,
  设 $p$ 对应 $\gp \subseteq A$, 则由 Hilbert 零点定理即知
  $k(p) = A / \gp$ 是 $k$ 的有限扩张.

  因此, 若 $X$ 是域 $k$ 上的有限型概形,
  由于每个仿射开集中有(相对的)闭点, 而由上述断言可知若 $p \in X$ 在某个开集中闭,
  就一定是闭点; 因此 $X$ 中的闭点稠密.

  若去除 $X$ 的有限型条件, 则任取离散赋值环 $A$,
  那么 $\Spec A$ 即不满足条件 (因为存在非幂零但属于所有极大理想的元素).
\end{proofc}

\begin{exercise}
  令 $X$ 为域 $k$ (不一定代数闭) 上有限型概形.
  \begin{enumerate}[label={(\alph*)}]
    \item 证明以下三个条件等价 (若它们成立, 则称 $X$\emph{几何不可约}):
          \begin{enumerate}[label={(\roman*)}]
            \item $X \times_k \bar{k}$ 不可约, 其中 $\bar{k}$ 表示 $k$ 的代数闭包.
            \item $X \times_k k_s$ 不可约, 其中 $k_s$ 表示 $k$ 的可分闭包.
            \item 对 $k$ 的任意扩域 $K$, $X \times_k K$ 都不可约.
          \end{enumerate}
    \item 证明以下三个条件等价 (若它们成立, 则称 $X$\emph{几何既约}):
          \begin{enumerate}[label={(\roman*)}]
            \item $X \times_k \bar{k}$ 既约, 其中 $\bar{k}$ 表示 $k$ 的代数闭包.
            \item $X \times_k k_p$ 既约, 其中 $k_p$ 表示 $k$ 的完美闭包.
            \item 对 $k$ 的任意扩域 $K$, $X \times_k K$ 都既约.
          \end{enumerate}
    \item 如果 $X \times_k \bar{k}$ 整, 就说 $X$ \emph{几何整}.
          给出一个既不几何不可约也不几何既约的整概形. % TODO: 不几何不可约?
  \end{enumerate}
\end{exercise}

\begin{proofc}
  \begin{enumerate}[label={(\alph*)}]
    \item 我们使用 Stacks Project 中的\Stacks{037K}:
          对于 $k$ 上的环 $R$, 若 $S \otimes_k k_p$ 的素谱不可约,
          则对任意扩域 $K / k$, $S \otimes_k K$ 的素谱不可约.
          也就是说 $X = \Spec R$ 时命题成立.

          对于任意的 $X$, 设其可以被仿射开集 $\{ U_i \}$ 覆盖,
          则对任意的 $K$, $X \times_k K$ 都可以被 $V_i = \pi^{-1}(U_i) \cong U_i \times_k K$ 覆盖.
          且对任意 $i, j$, 易知 $V_i \cap V_j \cong (U_i \cap U_j) \times_k K$
          非空当且仅当 $U_i \cap U_j$ 非空.

          而 $X \times_k K$ 不可约当且仅当每个 $V_i$ 都不可约并且 $V_i \cap V_j$ 都非空.
          因此即证.
    \item 类似(a): 我们使用\Stacks{030V}:
          对于 $k$ 上的环 $S$, 若 $S \otimes_k k_s$ 既约,
          则对任意扩域 $K / k$, 都有 $S \otimes_k K$ 既约.
          换言之, 当 $X$ 仿射时, 命题成立.

          当 $X$ 是任意概形时, 类似于 (a), 并且此时 $X$ 既约当且仅当其可以被既约开子概形覆盖.
          因此即证.
    \item 设 $k = \F_p(x), A = \F_q(x^{1/p}), X = \Spec A$,
          其中 $p$ 是素数, $q = p^2$.
          则 $A$ 是整环, 因此 $X$ 是整概形.

          那么 $X \times_k \F_p(x^{1/p}) = \Spec (A \otimes_k \F_p(x^{1/p}))$,
          而
          \[
          A \otimes_k \F_p(x^{1/p}) \cong \F_q(y, z) / (y^p - z^p)
          \cong \F_q(y, z) / ((y - z)^p)
          \]
          不既约, 因此 $X$ 不既约.

          而 $X \times_k \F_q = \Spec (A \otimes_k \F_q)$,
          其中 $A \otimes_k \F_q \cong (\F_q \otimes_k \F_q)(x) \cong (\F_q \oplus \F_q)(x)$ 并非不可约,
          因此 $X$ 并不几何不可约.
          \qedhere
  \end{enumerate}
\end{proofc}

\begin{exercise}[Noether 归纳法]
  设 $X$ 为 Noether 拓扑空间, 并令 $\mathcal{P}$ 是对 $X$ 的闭集定义的性质.
  假设对 $X$ 的任意闭子集 $Y$, 若 $Y$ 的所有真闭子集都满足 $\mathcal{P}$,
  则 $Y$ 也满足 $\mathcal{P}$ (特别地, 空集必定满足 $\mathcal{P}$).
  那么 $X$ 的所有闭子集都满足 $\mathcal{P}$.
\end{exercise}

\begin{proofc}
  反证. 若不然, 则由于 $X$ Noether, 存在极小的不满足 $\mathcal{P}$ 的闭子集, 矛盾.
\end{proofc}

\begin{exercise}[Zariski 空间]
  若拓扑空间 $X$ 是 Noether 空间, 且其任意非空不可约闭集都有唯一的一般点,
  则称其为\emph{Zariski 空间}.

  例如说, 令 $R$ 是离散赋值环, $T = \spp(\Spec R)$.
  则 $T$ 包含两个点 $t_0 = $ 极大理想, $t_1 = $ 零理想.
  其开集有 $\emptyset, \{ t_1 \}, T$.
  其是不可约 Zariski 空间, 具有一般点 $t_1$.

  \begin{enumerate}[label={(\alph*)}]
    \item 证明若 $X$ 是 Noether 概形, 则 $\spp(X)$ 是 Zariski 空间.
    \item 证明 Zariski 空间的每个极小的非空闭子集都是单点集. 我们将这些点称之为闭点.
    \item 证明 Zariski 空间满足 $T_0$ 公理: 任意两个点都可以用开集区分.
    \item 若 $X$ 是不可约 Zariski 空间, 则其一般点包含在任意非空开集中.
    \item 若 $x_0, x_1$ 是拓扑空间 $X$ 中的点, $x_0 \in \{x_1\}^-$,
          就称 $x_0$ 是 $x_1$ 的\emph{特殊化}, 记作 $x_1 \spto x_0$.
          我们也说 $x_1$ \emph{特殊化为} $x_0$, 以及 $x_1$ 是 $x_0$ 的\emph{一般化}.
          现设 $X$ 是 Zariski 空间.
          证明由特殊化定义的偏序 ($x_1 > x_0$ 当且仅当 $x_1 \spto x_0$)
          中的极小点就是 $X$ 的不可约分支的一般点.
          证明闭集包含其所有点的特殊化 (即\emph{对特殊化稳定}).
          同理, 开集\emph{对一般化稳定}.
    \item 令 $t$ 是命题 (2.6) 中定义的拓扑空间的函子.%
          \footnote{
          若 $X$ 是任意拓扑空间, 则 $t(X)$ 是 $X$ 的所有不可约闭集构成的集合,
          $t(X)$ 中的闭集形如 $t(Y)$, 其中 $Y$ 是 $X$ 的闭集.}%
          若 $X$ 是 Noether 空间, 证明 $t(X)$ 是 Zariski 空间.
          进一步地, $X$ 是 Zariski 空间当且仅当 $\alpha \colon X \to t(X)$ 是同胚.
  \end{enumerate}
\end{exercise}
\printfootnotes
\clearfootnotes

\begin{proofc}
  \begin{enumerate}[label={(\alph*)}]
    \item 概形的不可约闭集都有一般点, 因此是 Zariski 空间.
    \item 设 $X$ 是 Zariski 空间, $Z$ 是其极小非空闭子集.
          则 $Z$ 中所有点都是 $Z$ 的一般点, 因此由一般点的唯一性即知 $Z$ 是单点集.
    \item 若 $x, y$ 不能被区分, 则他们有一样的闭包, 这个闭包以这两个点为一般点, 矛盾.
    \item 平凡.
    \item 平凡.
    \item $t$ 给出了 $X$ 的闭集到 $t(X)$ 的闭集的双射,
          因此若 $X$ Noether, 则 $t(X)$ 也 Noether,
          此时 $t(X)$ 按定义当然是 Zariski 空间.

          若 $X$ 是 Zariski 空间, 则 $\alpha$ 是闭的连续双射, 所以是同胚.
          反过来若 $\alpha$ 是同胚, 则 $X$ 当然是 Zariski 空间.
          \qedhere
  \end{enumerate}
\end{proofc}

\begin{exercise}[可构造集]
  令 $X$ 为 Zariski 空间.
  记 $\mathcal{F}$ 是包含 $X$ 的所有闭集且在有限交和取补集下封闭的最小集族.
  我们将 $\mathcal{F}$ 中的集合称为 $X$ 的\emph{可构造子集}.

  \begin{enumerate}[label={(\alph*)}]
    \item $X$ 中的开集与闭集的交集称为\emph{局部闭集}.
          证明 $X$ 的一个子集可构造当且仅当它可以写成局部闭集的有限不交并.
    \item 证明 $X$ 中的某个可构造集稠密当且仅当它包含一般点.
          进一步地, 此时它一定包含某个非空开集.
    \item $X$ 的子集 $S$ 是闭集当且仅当它可构造并且对特殊化封闭.
          类似地, 子集 $T$ 是开集当且仅当它可构造并且对一般化封闭.
    \item 若 $f \colon X \to Y$ 是 Zariski 空间之间的连续映射,
          则 $Y$ 的任意可构造子集的原像也是 $X$ 的可构造子集.
  \end{enumerate}
\end{exercise}

\begin{proofc}
  \begin{enumerate}[label={(\alph*)}]
    \item 集合 $S \subseteq X$ 可以写成局部闭集的有限不交并,
          当且仅当它可以写成局部闭集的有限并,
          当且仅当它可以由开集和闭集做有限次交和并操作得到,
          当且仅当它属于 $\mathcal{F}$.
    \item 若 $S \subseteq X$ 是可构造的稠密集,
          由 (a), $S = \bigcup_{i = 1}^n U_i \cap Z_i$,
          其中 $U_i$ 是开集, $Z_i$ 是闭集.
          那么 $X = \bar{S} = \bigcup_{i = 1}^n \overline{U_i} \cap Z_i$.
          因此存在 $i$ 使得 $\overline{U_i} \cap Z_i = X$, 即 $Z_i = X$ 且 $U_i$ 在 $X$ 中稠密.
          因此 $X$ 的一般点 $\in U_i \subseteq S$.
          此时 $S$ 包含非空开集 $U_i$.

          反过来若 $S$ 包含一般点, 则 $S$ 稠密.
    \item 设 $S$ 可构造并且对特殊化封闭.
          设 $Z$ 是 $\bar{S}$ 中的不可约闭集.
          由 (b), $S$ 包含 $Z$ 中的一般点, 因此由其对特殊化封闭即知 $Z \subseteq S$.
          所以 $\bar{S} \subseteq S$, 从而 $S$ 是闭集.

          反过来, 闭集当然可构造且对特殊化封闭.
    \item 平凡. 因为连续映射的原像保持闭集, 有限交和补集.
          \qedhere
  \end{enumerate}
\end{proofc}

\begin{exercise}
  可构造集的重要性由下述的 Chevally 定理给出:
  设 $f \colon X \to Y$ 是 Noether 概形之间的有限型同态.
  那么 $X$ 的任意构造集的像仍是构造集.
  特别地, $f(X)$ 不一定是开集或者闭集, 但是一定是可构造集.
  请按如下步骤证明该定理.
  \begin{enumerate}[label={(\alph*)}]
    \item 归约到在 $X, Y$ 都是整的 Noether 仿射概形且 $f$ 支配的情况下,
          证明 $f(X)$ 本身可构造.
    \item 这种情况下, 通过如下交换代数结果证明 $f(X)$ 包含 $Y$ 的非空开子集:

          若 $A \subseteq B$ 分别是 Noether 整环, 且 $B$ 是有限生成 $A$-代数.
          那么对任意 $b \neq 0 \in B$, 存在 $a \in A$, 满足:
          对任意将 $A$ 映射到某个代数闭域 $K$ 中的同态 $\varphi \colon A \to K$,
          只要 $\varphi(a) \neq 0$, 就可以将其延拓为 $\varphi' \colon B \to K$,
          使得 $\varphi'(b) \neq 0$.
          [\emph{提示: 通过对 $B$ 的生成元个数做归纳来证明这个代数结果.
          然后使用 $b = 1$ 的情况.}]
    \item 通过对 $Y$ 做 Noether 归纳来完成证明.
    \item 给出一个如下的例子: $f \colon X \to Y$ 是代数闭域上的代数簇之间的态射,
          而 $f(X)$ 不开也不闭.
  \end{enumerate}
\end{exercise}

\begin{proofc}
  \begin{enumerate}[label={(\alph*)}]
    \item 设 $S \subseteq X$ 是构造集,
          不妨设 $S = \bigcup U_i \cap Z_i$, 其中 $U_i$ 是开集, $Z_i$ 是闭集.
          则只需要对每个 $i$, 证明 $f(U_i \cap Z_i)$ 可构造.
          由于 $X$ Noether, $U_i$ 可以写作有限多个仿射开集 $V_{ij}$ 的并.
          只需证明 $f(V_{ij} \cap Z_i)$ 可构造.
          而 $V_{ij} \cap Z_i$ 可以看作仿射概形 $V_{ij}$ 的闭子概形.

          这样, 我们就归约到了 $X$ 本身是仿射概形, 且只需证明 $f(X)$ 可构造的情况.
          同理也可以归约到 $X, Y$ 都仿射的情况.
          再通过把 $X, Y$ 替换为 $X_{\red}, Y_{\red}$ 并取不可约分支, 即可假设 $X, Y$ 整.
    \item 先证明这个代数结论.
          通过对 $B$ 在 $A$ 上的生成元个数归纳,
          只需要考虑 $B = A[x]$ 或者 $B = A[x] / (f(x))$ 的情况,
          其中 $f$ 是首一不可约多项式.
          设 $b = b(x) \in A[x]$, 而 $b(x)c(x) \equiv a \pmod{f(x)}$,
          其中 $a \in A, c \in A[x]$. 则此 $a$ 满足条件.
          这样的 $a, c(x)$ 的存在性可以由裴属定理保证,
          或者说考虑 $\frac{c}{a} = b^{-1} \in (\Frac A)[x] / (f(x))$.

          接下来设 $X = \Spec B, Y = \Spec A$.
          由于 $f$ 支配, 对应的映射 $A \to B$ 是单射.
          取 $b = 1$, 则存在 $a \in A$ 满足上述命题条件.
          此时若 $a \notin \gp \subset A$, 则 $A / \gp$ 可以嵌入到某个代数闭域 $K$ (其分式域的代数闭包),
          因此给出了映射 $\varphi \colon A \to K$, 且 $\varphi(a) \neq 0$.
          由上述命题, 此时 $\varphi$ 可以延拓为 $\varphi' \colon B \to K$.
          因此 $f(\ker \varphi') \cap A = \ker \varphi = \gp$,
          从而 $D(a) \subseteq f(X)$.
    \item 由类似于 (a) 的方法, 可以将 (b) 推广为: 只要 $f \colon X \to Y$ 有限型,
          $f(X)$ 就包含一个非空开集.

          我们用 Noether 归纳法证明: 对 $Y$ 中每个闭子集 $E$, $f(X) \cap E$ 可构造.
          若 $E$ 的每个真闭子集都具有此性质, 则考虑 $Z = f^{-1}(E)$ (配备任意闭子概形结构),
          则 $f(X) \supseteq f(Z)$ 包含 $E$ 中的非空开集 $U$.
          因此由归纳假设, $f(X) \cap E = U \cup (f(X) \cap (E \setminus U))$ 可构造.
    \item 由去掉原点的直线到平面的嵌入的像不开也不闭.
          \qedhere
  \end{enumerate}
\end{proofc}

\begin{exercise}[维数]
  令 $X$ 为域 $k$ (未必代数闭) 上的有限型整概形. 利用 (I, §1) 中的结果%
  \footnote{若 $B$ 是域 $B$ 上的有限生成整环, 则 $\dim B = \trd K(B) / k$,
    且对任意素理想 $\gp$ 有 $\height \gp + \dim B / \gp = \dim B$.}%
  证明下列命题:
  \begin{enumerate}[label={(\alph*)}]
    \item 对于 $X$ 中的闭点 $P$, 有 $\dim X = \dim \sO_P$. 对于环, $\dim$ 总是表示 Krull 维数.
    \item 令 $K(X)$ 表示 $X$ 的函数域, 证明 $\dim X = \trd K(X) / k$.
    \item 若 $Y$ 是 $X$ 的闭子集, 则 $\codim(Y, X) = \inf \{ \dim \sO_{P, X} \mid P \in Y \}$.
    \item 若 $Y$ 是 $X$ 的闭子集, 则 $\dim Y + \codim(Y, X) = \dim X$.
    \item 若 $U$ 是 $X$ 的非空开子集, 则 $\dim U = \dim X$.
    \item 若 $k' / k$ 是域扩张, 则 $X' = X \times_k k'$ 的每个不可约分支都具有和 $X$ 相同的维数.
  \end{enumerate}
\end{exercise}
\printfootnotes
\clearfootnotes

\begin{proofc}
  \begin{enumerate}[label={(\alph*)}]
    \item 首先考虑到 $\dim X = \sup \{ n \mid V_0 \subsetneq V_1 \subsetneq \dots \subsetneq V_n \}$,
          其中 $V_k$ 都是 $X$ 中的不可约闭集.
          取闭集的一般点即知 $\dim X = \sup \{ n \mid P_0 \spto P_1 \spto \dots \spto P_n \}$.
          而对于点 $P$, 易知有 $\dim \sO_P = \sup \{ n \mid P_0 \spto P_1 \spto \dots \spto P_n = P \}$.
          因此 $\dim X = \sup_{P \in X} \dim \sO_P$.
          但对 $X$ 中任意仿射开集 $\Spec A$, 有 $\dim A = \trd \Frac(A) / k = \trd K(X) / k$,
          且对于 $A$ 中任意极大理想 $\gm$ 有 $\dim A_{\gm} = \height \gm = \dim A$.
          因此对任意闭点 $P$, $\dim X = \dim \sO_P = \trd K(X) / k$.
    \item 在 a 中已证.
    \item 由定义, $\codim(Y, X) = \inf_{V \text{不可约} \subseteq Y} \codim(V, X)$.
          取一般点即证.
    \item 对任意点 $P$, 设其有仿射邻域 $\Spec A$, 在 $A$ 中对应素理想 $\gp$.
          则由 $\height \gp + \dim A / \gp = \dim A$ 即知
          $\dim (\overline{\{P\}} \cap A) + \dim \sO_{P, X} = \dim A = \dim X$.
          由于这对任意仿射邻域成立, 即知 $\dim \overline{\{P\}} + \dim \sO_{P, X} = \dim X$.
          再由 c 即证.
    \item $U$ 必定包含某个仿射开集, 而仿射开集与 $X$ 具有相同维数.
    \item 由 e, 仅需考虑仿射开集情况.
          此时利用 $\dim A = \trd \Frac(A) / k$ 易证.
          \qedhere
  \end{enumerate}
\end{proofc}

\begin{exercise}
  令 $R$ 为离散赋值环, 且包含其剩余域 $k$.
  令 $X = \Spec R[t]$ 为 $\Spec R$ 上的仿射直线.
  证明习题 3.20 中的 (a), (d), (e) 对 $X$ 不成立.
\end{exercise}

\begin{proofc}
  设 $R$ 的素理想为 $(0), \gm = (\varpi)$.
  则 $R[t]$ 的素理想有 $(0), (f), (\varpi, g)$. 其中 $f$ 是 $R$ 上的不可约多项式,
  $g$ 是 $k$ 上的不可约多项式. 因此 $\dim X = \dim R[t] = 2$.

  然而 $R[t]$ 中, $(t\varphi - 1)$ 也是极大理想
  (因 $R[t] / (t\varphi - 1) \cong R_{\varphi}$ 是域),
  但其局部环维数为 $1$. 因此 a 不成立.

  取 $Y = \{ [(t\varphi - 1)] \}$, 则 d 不成立.

  取 $U = D(\varphi)$, 则 $\sO|_U \cong \Spec F[t]$, 其中 $F = \Frac(R)$.
  因此 $\dim U = 1$, e 不成立.
\end{proofc}

\begin{exercise}[态射纤维的维数]
  令 $f \colon X \to Y$ 为 $k$ 上有限型整概形之间的支配映射.
  \begin{enumerate}[label={(\alph*)}]
    \item 令 $Y'$ 为 $Y$ 中的不可约闭子集, 其一般点 $\eta'$ 属于 $f(X)$.
          令 $Z$ 为 $f^{-1}(Y')$ 的不可约分支, 且 $\eta' \in f(Z)$,
          证明 $\codim(Z, X) \leq \codim(Y', Y)$.
    \item 令 $e = \dim X - \dim Y$ 为 $X$ 在 $Y$ 上的\emph{相对维数}.
          对任意 $y \in f(X)$, 证明纤维 $X_y$ 的不可约分支的维数不小于 $e$.
          [\emph{提示: 令 $Y' = \{y\}^-$, 使用 (a) 和习题 3.20 (b).}]
    \item 证明存在 $X$ 的稠密开子集 $U$, 使得对任意 $y \in f(U)$,
          都有 $\dim U_y = e$.
          [\emph{提示: 首先规约到 $X, Y$ 都仿射的情况. 记 $X = \Spec A, Y = \Spec B$,
          则 $A$ 是有限生成 $B$ 代数. 取 $K(A)$ 在 $K(B)$ 上的一组超越基
          $t_1, \dots, t_e \in A$, 令 $X_1 = \Spec B[t_1, \dots, t_e]$.
          则 $X_1$ 同构于 $Y$ 上的 $e$ 维仿射空间,
          且映射 $X \to X_1$ 一般有限. 使用习题 3.7.}]
    \item 回到原本的映射 $f \colon X \to Y$.
          对任意整数 $h$, 令 $E_h \subseteq X$ 为满足下述条件的点 $x$ 构成的子集:
          记 $y = f(x)$, 存在 $X_y$ 的包含 $x$ 的不可约分支 $Z$, 且 $\dim Z \geq h$.

          证明:
          \begin{enumerate}[label={\arabic*.}]
            \item $E_e = X$ (使用上面的 (b));
            \item 若 $h > e$, 则 $E_h$ 在 $X$ 中不稠密.
            \item 对任意 $h$, $E_h$ 闭 (对 $\dim X$ 做归纳).
          \end{enumerate}
    \item 证明如下的 Chevalley 定理 (见 Cartan, Chevalley [1, exposé 8]):
          对任意整数 $h$, 令 $C_h = \{ y \in Y \mid \dim X_y = h \}$.
          则 $C_h$ 可构造, 且 $C_e$ 包含 $Y$ 的某个稠密开子集.
  \end{enumerate}
\end{exercise}

\begin{proofc}
  \begin{enumerate}[label={(\alph*)}]
    \item 按定义, $\codim(Z, X) =$ 以 $Z$ 的一般点 $\theta$ 结尾的最长特殊化序列长度,
          $\codim(Y', Y)$ 同理. 而 $f$ 保持特殊化\emph{(连续映射都保持特殊化)},
          且 $\eta' \in f(Z) \implies f(\theta) \spto \eta'$. 因此即证.
    \item 作为拓扑空间, $X_y$ 同胚于 $f^{-1}(y)$.
          因此若 $Z$ 是 $X_y$ 的不可约分支,
          则 $Z$ 同胚于 $\overline{f^{-1}(y)}$ 的某个不可约分支 $Z'$
          与 $f^{-1}(y)$ 的交集. 又由 3.20 (e),
          \[
          \dim Z = \dim (Z' \cap f^{-1}(y))
          = \dim \overline{Z' \cap f^{-1}(y)} = \dim Z'.
          \]
          而由 (a) 与 3.20 (d), 取 $Y' = \overline{\{y\}}$ 得
          \[
          \dim Z' = \dim X - \codim(Z', X) \geq e + \dim Y - \codim(Y', Y)
          = e + \dim Y' \geq e.
          \]
    \item 任取 $Y$ 中仿射开集和其原像中仿射开集以替换 $Y$ 和 $X$,
          则 $f$ 仍然支配, 且只要这种情况成立, 原命题显然成立.
          记 $X = \Spec A, Y = \Spec B$.

          此时由 3.20 (b),
          $e = \trd \Frac A / k - \trd \Frac B / k = \trd \Frac A / \Frac B$.
          按提示, 取 $t_1, \dots t_e \in A$ 使得它们是 $\Frac(A)$ 在 $\Frac(B)$
          上的超越基.
          令 $X_1 = \Spec B[t_1, \dots t_e]$, 则 $X_1$ 是 $Y$ 上的 $e$ 维仿射空间
          且 $g \colon X \to X_1$ 一般有限 ($X_1$ 的一般点的原像恰仅包含 $X$ 的一般点).
          由 3.7, 存在 $X_1$ 中的稠密开集 $V$ 使得 $U = g^{-1}(V) \to V$ 有限.
          那么若 $y \in f(U)$, 则 $U_y$ 在 $V_y$ 上有限且支配;
          而 $V_y$ 是 $\Spec k(y)[t_1, \dots t_e]$ 中的稠密开集,
          因此 $\dim U_y = \dim V_y = e$.
    \item \begin{enumerate}[label={\arabic*.}]
            \item 由 (b), $E_e = X$.
            \item 由 (c), 若 $h > e$, 则 $E_h \subseteq X \setminus U$.
                  因此 $E_h$ 不稠密.
            \item 我们对 $\dim X$ 做归纳.

                  若 $h \leq e$, 则由 (a) 已证完.
                  否则以 (c) 的方法取出一开集 $U \subseteq X$,
                  则 $U \cap E_h = \emptyset$.
                  由于 $X$ Noether, $X \setminus U$
                  有有限个不可约分支 $\{X^j\}$, 只需证明对其中每个不可约分支,
                  都有 $X^j \cap E_h$ 在 $X^j$ 中闭.

                  我们赋予 $X^j$ 既约诱导闭子概形结构,
                  记 $Y^j = \overline{f(X^j)}$, 也赋予既约诱导闭子概形结构.
                  那么 $X^j, Y^j$ 也是 $k$ 上有限型整概形,
                  且 $f|_{X^j} : X^j \to Y^j$ 支配.

                  设把 $X, Y$ 替换为 $X^j, Y^j$ 后定义出的 $E_h$ 为 $E^j_h$,
                  原本的 $E_h$ 仍记为 $E_h$.
                  显然有 $E^j_h \subseteq X^j \cap E_h$.

                  另一方面, 若 $x \in X^j \cap E_h$, 记 $y = f(x)$,
                  则存在 $X_y$ 的包含 $x$ 的不可约分支 $Z$ 具有维数 $h$.
                  若 $y \notin U$, 则 $X_y$ 是若干个 $X^k_y$ 的并,
                  因此不可约分支 $Z$ 是 $X^j_y$ 的子集, 从而 $x \in E^j_h$.
                  即使 $y \in U$, 也一定有 $Z \cap U_y = \emptyset$,
                  否则 $\dim Z = \dim (Z \cap U_y) \leq \dim U_y = e$.
                  接下来同上即知 $x \in E^j_h$.

                  因此我们知道 $E_h \cap X^j = E^j_h$. 由于 $\dim X^j < \dim X$,
                  按归纳假设知 $E_h \cap X^j$ 闭.
                  因此由数学归纳法, 结论成立.

                  [\emph{这里没有归纳起点, 因为 $\dim X = 0$ 的时候 $U = X = $ 单点集.}]
          \end{enumerate}
    \item 由于 $E_h$ 闭, 由 3.19, $C_h = f(E_h) \setminus f(E_{h + 1})$ 可构造.
          而 $Y$ 的一般点 $\in C_e$, 因此 $C_e$ 稠密.
          由 3.18 (b), $C_e$ 包含某个稠密开集.
          \qedhere
  \end{enumerate}
\end{proofc}

\begin{exercise}
  设 $V, W$ 是代数闭域 $k$ 上的概形, $V \times W$ 是其乘积 (在 I, 习题 3.15, 3.16 中定义),
  $t$ 是 2.6 中的函子, 则 $t(V \times W) = t(V) \times_{\Spec k} t(W)$.
\end{exercise}

\end{document}
