\documentclass{article}
\usepackage{xeCJK}
\usepackage{fontspec}
\usepackage{amsmath}
\usepackage{amsthm}
\usepackage{unicode-math}
\setmathfont{Libertinus Math}
\usepackage[a4paper, margin=2cm]{geometry}
\usepackage{enumitem}
\usepackage{tikz-cd}
\usepackage{hyperref}
\hypersetup{
  colorlinks = true,
  linkcolor = blue
}

\def\proofname{证明}

\newtheoremstyle{exercise}%
{3pt}{3pt}
{}%
{}%
{\bfseries}%
{.}%
{.2em}%
{}
\theoremstyle{exercise}
\newtheorem{exercise}{习题}[section]
\newenvironment{proofc}{\proof}{\endproof}
\def\printfootnotes{}

%--------- copy from btex ---------%

\def\ga{\mathfrak{a}}
\def\gp{\mathfrak{p}}
\def\gq{\mathfrak{q}}
\def\gm{\mathfrak{m}}
\def\gn{\mathfrak{n}}
\def\uA{\mathbf{A}}
\def\Z{\mathbb{Z}}
\def\F{\mathbb{F}}
\def\Q{\mathbb{Q}}
\def\R{\mathbb{R}}
\def\C{\mathbb{C}}
\def\id{\mathrm{id}}
\def\cO{\mathscr{O}}
\def\Sch{\mathfrak{Sch}}
\def\Rings{\mathfrak{Rings}}
\def\spto{\rightsquigarrow}
\def\red{\mathrm{red}}
\def\spp{\operatorname{sp}}
\def\Hom{\operatorname{Hom}}
\def\Spec{\operatorname{Spec}}
\def\Proj{\operatorname{Proj}}
\def\Supp{\operatorname{Supp}}
\def\coker{\operatorname{coker}}
\def\im{\operatorname{im}}
\def\dim{\operatorname{dim}}
\def\codim{\operatorname{codim}}
\def\height{\operatorname{height}}
\def\Frac{\operatorname{Frac}}
\def\trd{\operatorname{tr.d.}}
\def\leq{\leqslant}
\def\geq{\geqslant}

\begin{document}

这里是 Hartshorne 第二章第二节习题.

\setcounter{section}{1}

\section{概形}

\begin{exercise}
  设 $A$ 是环, $X = \Spec A$, $f \in A$, $D(f) \subseteq X$ 为 $V((f))$ 的开补集.
  证明 $(D(f), \cO_X|_{D(f)})$ 同构于 $\Spec A_f$.
\end{exercise}

\begin{proofc}
  构造同构映射 $\phi \colon D(f) \to \Spec X$ 如下.
  若 $\gp \in D(f)$, 则 $f \notin \gp$, 从而 $\gp_f \subseteq A_f$.
  定义 $\phi(\gp) = \gp_f$, 则 $\phi$ 作为集合的映射是双射.
  此外, 还有自然的同构 $A_{\gp} \cong (A_f)_{\gp_f}$,
  因此容易定义并验证同构 $(\phi, \phi^\sharp) \colon (D(f), \cO_X|_{D(f)}) \cong \Spec A_f$.
\end{proofc}

\begin{exercise}
  设 $(X, \cO_X)$ 是概形, $U \subseteq X$ 是开集. 证明 $(U, \cO_X|_U)$ 是概形.
  我们将其称之为开集 $U$ 上的\emph{诱导概形结构},
  并将 $(U, \cO_X|_U)$ 称为 $X$ 的\emph{开子概形}.
\end{exercise}

\begin{proofc}
  注意到任意仿射概形中存在一组由同构于仿射概形的开集构成的基,
  即 $(D(f), \cO|_{D(f)})$. 因此仿射性是局部性质.
  从而概形的开子集仍然局部仿射, 因此也是概形.
\end{proofc}

\begin{exercise}
  设 $(X, \cO_X)$ 是概形. 如果对任意开集 $U \subseteq X$, $\cO_X(U)$ 中都没有幂零元素,
  就说 $(X, \cO_X)$ \emph{既约}.
  \begin{enumerate}[label={(\alph*)}]
    \item 证明:$(X, \cO_X)$ 既约当且仅当对任意 $P \in X$, 局部环 $\cO_{X, P}$ 里都没有幂零元素.
    \item 令 $(X, \cO_X)$ 为概形. 令 $(\cO_X)_{\red}$ 为预层 $U \mapsto \cO_X(U)_{\red}$,
          其中, 对任意环 $A$, $A_{\red}$ 表示 $A$ 商掉幂零元的理想构成的商环.
          证明 $(X, (\cO_X)_{\red})$ 是概形.
          我们称之为 $X$ 的\emph{既约化概形}, 记作 $X_{\red}$.
          证明存在同态 $X_{\red} \to X$, 其限制在底空间上是同胚.
    \item 设 $f \colon X \to Y$ 是概形同态, 且 $X$ 既约. 证明 $f$ 穿过 $Y_{\red}$.
  \end{enumerate}
\end{exercise}

\begin{proofc}
  \begin{enumerate}[label={(\alph*)}]
    \item 若 $U$ 是开集, $s \in \cO_X(U)$ 幂零, 则对任意 $P \in U$, $s_P$ 幂零.
          因此若 $\cO_{X, P}$ 中都无幂零元素, 即知 $s_P = 0 \forall P \in U$, 因此 $s = 0$.
          充分性即证.
          反之, 若 $X$ 既约, 则对任意 $P \in X$,
          $\cO_{X, P} = \varinjlim_{P \in U} \cO_X(U)$ 是既约环的直极限, 因此也既约.

    \item 只需注意到 $(A_f)_{\red} \cong (A_{\red})_f$,
          因此 $(\Spec A)_{\red} \cong \Spec A_{\red}$.
          $X_{\red} \to X$ 的映射容易构造, 即在每个截面上对应商同态.

    \item 若 $X$ 既约, 则同态 $f^\sharp \colon \cO_Y(U) \to f_*(\cO_X)(U)$
          必定将幂零元映射到 $0$, 因此穿过 $\cO_Y(U)_{\red}$.
          从而易知 $f$ 穿过 $Y_{\red}$.
          \qedhere
  \end{enumerate}
\end{proofc}

\begin{exercise}
  令 $A$ 是环, $(X, \cO_X)$ 是概形. 任意给定 $f \colon X \to \Spec A$,
  考虑整体截面上的映射, 就得到环同态 $A \to \Gamma(X, \cO_X)$.
  因此存在自然映射
  \[
    \alpha \colon \Hom_{\Sch}(X, \Spec A) \to \Hom_{\Rings}(A, \Gamma(X, \cO_X)).
  \]
  证明 $\alpha$ 是双射.
\end{exercise}

\begin{proofc}
  设 $g \colon A \to \Gamma(X, \cO_X)$, 定义映射 $f \colon X \to \Spec A$ 如下.
  对于 $P \in X$, 考虑复合映射 $A \to \Gamma(X, \cO_X) \to \cO_{X, P}$,
  则 $\cO_{X, P}$ 的极大理想的原像也是 $A$ 中的素理想, 即为 $f(P)$.
  由定义, 自然有映射 $f^\sharp_{\gp} \colon A_{\gp} \to f_*(\cO_X)_{\gp}$.
  因此容易定义概形同态 $(f, f^\sharp) \colon X \to \Spec A$.

  若记以上构造为自然映射
  $\beta \colon \Hom_{\Rings}(A, \Gamma(X, \cO_X)) \to \Hom_{\Sch}(X, \Spec A)$,
  不难验证 $\alpha$ 与 $\beta$ 互为逆映射. 从而 $\alpha$ 必定是双射.
\end{proofc}

\begin{exercise}
  描述 $\Spec \Z$, 并证明它是概形范畴中的终对象.
\end{exercise}

\begin{proofc}
  $\Spec \Z$ 的底空间是以所有素数为点的有限补空间.
  对一个开集 $U$, 设 $U$ 不包含的素数为 $p_1, \dots, p_k$,
  则 $\Gamma(U, \Spec \Z)$ 是所有分母仅有 $p_1, \dots, p_k$ 这些素因子的有理数构成的环.

  由习题 2.4 即知 $\Spec \Z$ 是概形范畴的终对象,
  因为 $\Z$ 是环范畴的始对象.
\end{proofc}

\begin{exercise}
  描述零环的谱, 并证明它是概形范畴的始对象.
\end{exercise}

\begin{proofc}
  零环的谱是空集. 显然是始对象.
\end{proofc}

\begin{exercise}
  令 $X$ 是概形. 对任意 $x \in X$, 设 $\cO_x$ 是 $x$ 处的局部环,
  $\gm_x$ 是其极大理想. 定义 $x$ 处的\emph{剩余域}是 $k(x) = \cO_x / \gm_x$.
  设 $K$ 是域. 证明要给出 $\Spec K \to X$ 的同态, 等价于给出点 $x \in X$
  及域嵌入 $k(x) \to K$.
\end{exercise}

\begin{proofc}
  $\Spec K$ 是单点空间, 因此由定义立证.
\end{proofc}

\begin{exercise}
  设 $X$ 是概形. 对 $x \in X$, 定义 $X$ 中 $x$ 处的\emph{Zariski 切空间} $T_x$
  是 $k(x)$-向量空间 $\gm_x / \gm_x^2$ 的对偶空间.
  假设 $X$ 是域 $k$ 上的概形, $k[\epsilon] / \epsilon^2$ 是 $k$ 上的\emph{对偶数环}.
  证明要给出从 $\Spec k[\epsilon]/\epsilon^2$ 到 $X$ 的同态,
  等价于给出一个\emph{$k$-有理点} $x \in X$ (即 $k(x) = k$) 和 $T_x$ 的一个元素.
\end{exercise}

\begin{proofc}
  $\Spec k[\epsilon] / \epsilon^2$ 也是单点空间. 因此由定义易证.
\end{proofc}

\begin{exercise}
  设 $X$ 是拓扑空间, $Z$ 是其不可约闭子集. $Z$ 的\emph{一般点} 就是闭包等于 $Z$ 的点.
  若 $X$ 是概形, 证明每个 (非空) 不可约闭子集都有唯一的一般点.
\end{exercise}

\begin{proofc}
  在一般情况下, 对任意与 $Z$ 相交的仿射开子集 $U$, 由上述推导即知
  存在唯一的 $\xi_U \in Z \cap U$ 使得 $\{ \xi_U \}^- \cap U = Z \cap U$.
  若 $U, V$ 是两个这样的开集, 则由不可约性质知 $U \cap V \cap Z$ 非空.
  取仿射开集 $W \subseteq U \cap V$ 使得 $W \cap Z$ 非空.
  由上述推导, $\xi_U$ 和 $\xi_V$ 也同时属于 $W$, 并且是 $W$ 中 $W \cap Z$ 的唯一一般点.
  因此所有 $\xi_U$ 全部相等, 也就是 $Z$ 的一般点.

  若 $X \cong \Spec A$ 是仿射概形, 则其非空不可约闭子集必定形如 $V(\gp)$,
  从而有唯一的一般点 $\gp$.
  进一步地, 若 $D(f)$ 是与 $V(\gp)$ 相交的仿射开集, 则 $\gp \in D(f)$,
  因此 $\gp$ 也是 $D(f) \cap V(\gp)$ 的一般点.
\end{proofc}

\begin{exercise}
  描述 $\Spec \R[x]$. 其底空间与 $\R$ 这个集合有何区别?与 $\C$ 呢?
\end{exercise}

\begin{proofc}
  $\Spec \R[x]$ 中有一般点 $(0)$, 还有若干闭点;闭点与 $\R[x]$ 中的不可约多项式一一对应:
  即对每个 $r \in \R$, 有闭点 $(x - r)$;对任意 $b^2 - 4c < 0$, 有闭点 $(x^2 + bx + c)$.
  截面则与习题 2.5 类似.

  其底空间比集合 $\R$ 多出一般点以及二次多项式对应的闭点.
  而与 $\C$ 相比, 每个复数都与其复共轭等同起来了 (此外当然也多出了一般点).
\end{proofc}

\begin{exercise}
  令 $k = \F_p$ 是 $p$ 元有限域, 描述 $\Spec k[x]$. 其点处的剩余域是什么?
  给定一个域, $\Spec k[x]$ 中有多少以其为剩余域的点?
\end{exercise}

\begin{proofc}
  $\Spec k[x]$ 的点有一个一般点 $(0)$, 以及若干闭点, 与首一不可约多项式一一对应.
  $(0)$ 处的剩余域是分式域 $k(x)$.
  若 $f$ 是不可约多项式, 则 $(f)$ 处的多项式是 $k[x] / (f) \cong \F_q$,
  其中 $q = p^{\deg f}$.

  若给定 $k$ 的有限扩域 $\F_q, q = p^n$,
  则以其为剩余域的点的个数即为 $k[x]$ 中 $n$ 次首一不可约多项式的个数,
  由 Gauss 公式即为
  \[
    \frac{1}{n} \sum_{d \mid n} \mu\bigl(\frac{n}{d}\bigr) q^d. \qedhere
  \]
\end{proofc}

\begin{exercise}[粘接引理]
  结论很有用, 但是证明平凡. 不写了! % TODO: 可以把题抄一遍.
\end{exercise}

\begin{exercise}
  若拓扑空间 $X$ 的任意开覆盖都有子覆盖, 就称 $X$\emph{拟紧} (其实就是一般情况下提及的紧).
  \begin{enumerate}[label={(\alph*)}]
    \item 证明:拓扑空间 Noether 当且仅当其任意开子集拟紧.
    \item 若 $X$ 是仿射概形, 证明 $\spp(X)$ 拟紧, 但是一般并不 Noether.
          如果 $\spp(X)$ Noether, 就说 $X$ Noether.
    \item 若 $A$ 是 Noether 环, 证明 $\spp(\Spec A)$ 是 Noether 空间.
    \item 给出上一条的逆命题的一个反例, 即 $\spp(\Spec A)$ 是 Noether 空间, 但 $A$ 不 Noether.
  \end{enumerate}
\end{exercise}

\begin{proofc}
  \begin{enumerate}[label={(\alph*)}]
    \item 由定义平凡.
    \item 若 $\spp(\Spec A) \subseteq \bigcup_i U_i$,
          不妨设每个 $U_i$ 都是基本开集 $D(f_i)$.
          那么作为理想, $1 = \sum_i (f_i)$, 即存在有限个 $f_i$ 可以生成 $A$.
          因此对应的有限个 $D(f_i)$ 覆盖 $\Spec A$, 从而覆盖 $\spp(\Spec A)$.
    \item 若 $A$ 是 Noether 环, 则其理想满足升链条件,
          对应在 $\Spec A$ 中就说明其闭集满足降链条件.
          因此 $\Spec A$ 是 Noether 空间, $\spp(\Spec A)$ 作为其子空间也是 Noether 空间.
    \item 设 $A = k[x_1, x_2, \dots] / (x_1^2, x_2^2, \dots)$.
          记 $\gp = (x_1, x_2, \dots) \subseteq A$, 则 $A / \gp \cong k$,
          且 $\gp$ 中元素都幂零. 因此 $A$ 只有 $\gp$ 一个素理想, 从而 $\Spec A$ Noether.
          但是 $A$ 显然不 Noether.
          \qedhere
  \end{enumerate}
\end{proofc}

\begin{exercise}
  \begin{enumerate}[label={(\alph*)}]
    \item 设 $S$ 是分次环. 证明 $\Proj S = \emptyset$ 当且仅当 $S_+$ 中仅包含幂零元素.
    \item 设 $\varphi \colon S \to T$ 是分次环的分次同态 (即保持次数的同态).
          令 $U = \{ \gp \in \Proj T \mid \gp \not \supseteq \varphi(S_+) \}$.
          证明 $U$ 是 $\Proj T$ 的开子集,
          且 $\varphi$ 决定了一个自然同态 $f \colon U \to \Proj S$.
    \item 即使 $\varphi$ 不是同构, $f$ 也可能是.
          比如说, 设 $\varphi_d \colon S_d \to T_d$ 在 $d \geq d_0$ 的情况下都是同构,
          其中 $d_0$ 是非负整数. 证明 $U = \Proj T$ 并且 $f \colon \Proj T \to \Proj S$ 是同构.
    \item 设 $V$ 是射影簇, 其分次坐标环是 $S$. 证明 $t(V) \cong \Proj S$.
  \end{enumerate}
\end{exercise}

\begin{proofc}
  \begin{enumerate}[label={(\alph*)}]
    \item 若 $S_+$ 中不仅包含幂零元素,
          则考虑不包含某个非幂零元素及其幂的极大真齐次理想,
          不难证明其是齐次素理想.

          反之, 设 $S_+$ 中仅包含幂零元素, 则若 $\gp \subseteq S$ 是齐次素理想,
          则 $\gp \supseteq \sqrt{(0)} \supseteq S_+$.
          因此一切齐次素理想都包含 $S_+$, 从而 $\Proj S = \emptyset$.
    \item $U = \Proj T - V(\varphi(S_+))$ 当然是 $\Proj T$ 中的开集.
          若 $\gp \in U$, 可以定义 $f(\gp) = \ker (S \to T \to T / \gp) = \varphi^{-1}(\gp)$.
          而 $f^\sharp$ 可以由 $\varphi$ 诱导的局部环同态 $S_{(f(\gp))} \to T_{(\gp)}$ 定义.
    \item 若 $\varphi_d$ 在 $d \geq d_0$ 的情况下都是同构,
          则 $T / \varphi(S)$ 中次数大于 $0$ 的齐次元素都是幂零元.
          因此易知 $U = \Proj T$.

          为证明 $f$ 是同构, 只需证明 $\varphi$ 诱导的局部环同态
          $S_{(\varphi^{-1}\gp)} \to T_{(\gp)}$ 都是同构. 取元素验证其既单又满即可.
    \item 不会.
          \qedhere
  \end{enumerate}
\end{proofc}

\begin{exercise}
  不会代数簇, 不写了.
\end{exercise}

\begin{exercise}
  令 $X$ 是概形, $f \in \Gamma(X, \cO_X)$, 定义
  \[
    X_f = \{ x \in X \mid f_x \notin \gm_x \}.
  \]
  其中 $f_x \in \cO_x$ 是 $f$ 在 $x$ 处的茎, $\gm_x$ 是 $\cO_x$ 的极大理想.
  \begin{enumerate}[label={(\alph*)}]
    \item 设 $U = \Spec B$ 是 $X$ 中的仿射开集, $\bar{f} \in \Gamma(U, \cO_X|_U)$ 是 $f$ 的限制,
          证明 $U \cap X_f = D(\bar{f})$. 由此说明 $X_f$ 是开集.
    \item 假设 $X$ 拟紧. 令 $A = \Gamma(X, \cO_X)$, $a \in A$ 且 $a$ 限制在 $X_f$ 上消失.
          证明存在 $n > 0$, 使得 $f^n a = 0$ [提示:用仿射开集覆盖 $X$].
    \item 现在假设 $X$ 可以由有限个仿射开集 $U_i$ 覆盖, 且交集 $U_i \cap U_j$ 全都拟紧
          (比如说, $\spp(X)$ 是 Noether 空间时即满足此条件).
          令 $b \in \Gamma(X_f, \cO_{X_f})$. 证明对某个 $n > 0$, $f^n b$ 是 $A$ 中元素的限制.
    \item 沿用 (c) 中的假设, 证明 $\Gamma(X_f, \cO_{X_f}) \cong A_f$.
  \end{enumerate}
\end{exercise}

\begin{proofc}
  \begin{enumerate}[label={(\alph*)}]
    \item 若 $x \in U$, 则 $f_x = \bar{f}_x$. 因此显然.
    \item 先设 $X = \Spec A$ 是仿射开集. 则 $X_f = D(f), \cO_X|_{X_f} \cong \Spec A_f$.
          因此 $a$ 限制在 $X_f$ 上消失等价于存在 $n > 0$ 使得 $f^n a = 0$.

          在一般情况下, 由于 $X$ 可以由仿射开集覆盖, 而其拟紧, 从而其可以由有限个仿射开集覆盖,
          设为 $U_1, \dots, U_k$, 其中 $U_i \cong \Spec B_i$.
          记 $f, a$ 在 $U_i$ 上的限制为 $\bar{f}_i, \bar{a}_i \in B_i$.
          由上述推导, 对每个 $i$, 存在 $n_i$ 使得 $\bar{f}_i^{n_i} \bar{a}_i = 0$.
          取 $n$ 为 $n_i$ 中的最大值, 则由层的唯一性公理即知 $f^n a = 0$.
    \item 先设 $X = \Spec A$ 是仿射开集, 则 $b \in \Gamma(X_f, \cO_{X_f}) \cong A_f$,
          从而存在 $n$ 使得 $f^n b$ 是 $A$ 中元素的限制.

          一般情况下, 同 (b), 设 $X$ 可以由 $U_1, \dots, U_k$ 覆盖, $U_i \cong \Spec B_i$.
          同理定义 $\bar{f}_i \in \Gamma(U_i, \cO_X), \bar{b}_i \in \Gamma(U_i \cap X_f, \cO_X)$.
          则存在 $n$, 使得每个 $\bar{f}_i^n \bar{b}_i$ 是 $a_i \in A$ 的限制.
          此时对每一对 $i \neq j$, $a_i - a_j$ 在 $U_i \cap U_j \cap X_f$ 上的限制为 $0$.
          因此由 (b), 存在 $m_{ij}$ 使得 $f^{n_{ij}} (a_i - a_j)$ 在 $U_i \cap U_j$ 上限制为 $0$.
          取 $m$ 为 $m_{ij}$ 的最大值, 则 $\{ f^m a_i \}$ 彼此兼容,
          从而可以粘贴成 $t \in A$, 其在 $X_f$ 上的限制即是 $f^{n + m} b$.
    \item 显然 $f$ 在 $\Gamma(X_f, \cO_{X_f})$ 上可逆. 从而由 (b) (c) 易证.
          \qedhere
  \end{enumerate}
\end{proofc}

\begin{exercise}[仿射性的判别条件]
  \begin{enumerate}[label={(\alph*)}]
    \item 设 $f \colon X \to Y$ 是概形同态, 且 $Y$ 可以由若干开集 $U_i$ 覆盖,
          使得每个限制映射 $f^{-1}(U_i) \to U_i$ 是同构. 证明 $f$ 也是同构.
    \item 概形 $X$ 仿射当且仅当存在有限个元素 $f_1, \dots, f_r \in A = \Gamma(X, \cO_X)$,
          使得每个开集 $X_{f_i}$ 都仿射, 且 $(f_1, \dots, f_r) = A$
          [提示:使用前面的习题 2.4 和习题 2.16d].
  \end{enumerate}
\end{exercise}

\begin{proofc}
  \begin{enumerate}[label={(\alph*)}]
    \item 容易知道 $f$ 在底空间上是同胚. 且 $f$ 在茎上都是同构, 从而 $f$ 是同构.

    \item 由习题 2.16d 知道 $X_{f_i} \cong \Spec A_{f_i}$.
          用习题 2.4 的方法构造映射 $g \colon X \to \Spec A$.
          不难发现 $g$ 将 $X_{f_i}$ 映射到 $D(f_i)$,
          且映射 $g(X_{f_i}) \colon \cO_X(X_{f_i}) \to A_{f_i}$ 是同构.
          因此再由习题 2.4 就知道 $g|_{X_{f_i}}$ 即是同构 $X_{f_i} \cong \Spec A_{f_i}$.
          由 $(f_1, \dots, f_r) = A$ 即知 $D(f_i)$ 覆盖 $\Spec A$. 因此由 (a) 即证.
          \qedhere
  \end{enumerate}
\end{proofc}

\begin{exercise}
  \begin{enumerate}[label={(\alph*)}]
          本习题中, 我们将比较环同态的若干性质和其诱导的谱的同态的性质.
    \item 设 $A$ 是环, $X = \Spec A, f \in A$. 证明 $f$ 幂零当且仅当 $D(f)$ 为空.
    \item 令 $\varphi \colon A \to B$ 是环同态, $f \colon Y = \Spec B \to X = \Spec A$
          是诱导的仿射概形同态.
          证明 $\varphi$ 是单射当且仅当对应的层映射 $f^\sharp \colon \cO_X \to f_* \cO_Y$ 是单射.
          更进一步地, 证明这种情况下 $f$ 是\emph{支配} 的, 即 $f(Y)$ 在 $X$ 中稠密.
    \item 在同样的假设下, 证明:若 $\varphi$ 是满射, 则 $f$ 将 $Y$ 同胚到 $X$ 的闭子集,
          且 $f^\sharp$ 是满射.
    \item 证明 (c) 的逆命题, 即如果 $f$ 将 $Y$ 同胚到 $X$ 的闭子集,
          且 $f^\sharp$ 是满射, 则 $\varphi$ 是满射
          [提示:考虑 $X' = \Spec(A / \ker \varphi)$, 并使用 (b) 和 (c)].
  \end{enumerate}
\end{exercise}

\begin{proofc}
  \begin{enumerate}[label={(\alph*)}]
    \item 平凡.
    \item 若 $f^\sharp$ 是单射, 则 $f^\sharp(X) \colon \cO_X(X) = A \to f_*\cO_Y(X) = B$
          是单射, 即 $\varphi$ 是单射.

          反之, 若 $\varphi$ 是单射, 则对任意 $a \in A$,
          $A_a \to B_{\varphi(a)}$ 也是单射;即 $f^\sharp(D(a))$ 是单射.
          若 $U$ 是开集, $s \in \cO_X(U), f^\sharp(U)(s) = 0$,
          则 $s$ 限制在每个 $D(a) \subseteq U$ 上为 $0$.
          由于 $D(a)$ 构成一组基, 由层的唯一性公理即知 $s = 0$. 因此 $f^\sharp$ 是单射.

          并且若 $\varphi$ 是单射, 则对任意 $a \in A$, $a$ 不幂零, $\varphi(a)$ 也不幂零.
          因此 $B_{\varphi(a)}$ 非 $0$ 环, 即 $f^{-1}(D(a)) \neq \emptyset$.
          因此 $f(Y)$ 与所有开集相交非空, 即稠密.
    \item 设 $\varphi$ 是满射, 则 $B \cong A / \ker \varphi$,
          从而 $B$ 的素理想通过 $f$ 和 $A$ 中所有包含 $\ker \varphi$ 的素理想一一对应.
          因此 $f$ 将 $Y$ 同胚到 $V(\ker \varphi) \subseteq A$.
          且类似 (b), 若 $a \in A$, 则 $A_a \to B_{\varphi(a)}$ 是满射.
          从而 $f^\sharp$ 在一组开集基上的映射都为满射, 因此 $f^\sharp$ 是满射
          (因为茎上的映射都是满射).
    \item 定义 $X' = \Spec(A / \ker \varphi)$,
          则 $\varphi$ 分解为 $\pi \colon A \to A / \ker \varphi$
          和 $\varphi' \colon A / \ker \varphi \to B$.
          因此 $f$ 也分解为 $f' \colon Y \to X'$ 和 $p \colon X' \to X$.
          由于 $\varphi'$ 是单射, $f'(Y)$ 在 $X'$ 中稠密.
          然而 $X'$ (拓扑上) 可以看作 $X$ 的子空间,
          从而 $f'(Y)$ 是 $X'$ 的闭集, 因此 $f'(Y) = X'$.

          而 $f^\sharp \colon \cO_X \to p_*\cO_{X'} \to f_* \cO_Y$ 是满射,
          因此由 $p$ 是单射即知 $f^{\prime\sharp} \colon \cO_{X'} \to f'_*\cO_Y$ 是满射.
          而 $f^\sharp$ 又是单射, 因此是同构.
          $f'$ 也是同胚, 所以 $X' \cong Y$, 因此 $A / \ker \varphi \cong B$, 即 $\varphi$ 是满射.
          \qedhere
  \end{enumerate}
\end{proofc}

\begin{exercise}
  令 $A$ 是环, 证明下列条件彼此等价:
  \begin{enumerate}[label={\arabic*)}]
    \item $\Spec A$ 不连通.
    \item 存在非零元素 $e_1, e_2 \in A$ 使得 $e_1e_2 = 0, e_1^2 = e_1, e_2^2 = e_2, e_1 + e_2 = 1$
          (这样的元素称为\emph{正交幂等元}).
    \item $A$ 同构于两个非零环的直积.
  \end{enumerate}
\end{exercise}

\begin{proofc}
  若 (2) 成立,
  则 $\Spec A = D(e_1) \cup D(e_2), D(e_1) \cap D(e_2) = \emptyset$, 因此 (1) 成立.

  若 (3) 成立, 则两个直积因子中的单位元即是正交幂等元, 从而 (2) 成立.

  若 (1) 成立, 记 $\Spec A = U_1 \cup U_2, U_1 \cap U_2 = \emptyset$.
  设 $U_1 = V(\ga_1), U_2 = V(\ga_2)$, 其中 $\ga_1, \ga_2$ 是根理想.
  则 $\ga_1 \cap \ga_2 = 0, \ga_1 + \ga_2 = A$. 因此 $A = \ga_1 \times \ga_2$.
  从而 (3) 成立.
\end{proofc}

\end{document}
